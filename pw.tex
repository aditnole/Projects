
% Default to the notebook output style

    


% Inherit from the specified cell style.




    
\documentclass[11pt]{article}

    
    
    \usepackage[T1]{fontenc}
    % Nicer default font (+ math font) than Computer Modern for most use cases
    \usepackage{mathpazo}

    % Basic figure setup, for now with no caption control since it's done
    % automatically by Pandoc (which extracts ![](path) syntax from Markdown).
    \usepackage{graphicx}
    % We will generate all images so they have a width \maxwidth. This means
    % that they will get their normal width if they fit onto the page, but
    % are scaled down if they would overflow the margins.
    \makeatletter
    \def\maxwidth{\ifdim\Gin@nat@width>\linewidth\linewidth
    \else\Gin@nat@width\fi}
    \makeatother
    \let\Oldincludegraphics\includegraphics
    % Set max figure width to be 80% of text width, for now hardcoded.
    \renewcommand{\includegraphics}[1]{\Oldincludegraphics[width=.8\maxwidth]{#1}}
    % Ensure that by default, figures have no caption (until we provide a
    % proper Figure object with a Caption API and a way to capture that
    % in the conversion process - todo).
    \usepackage{caption}
    \DeclareCaptionLabelFormat{nolabel}{}
    \captionsetup{labelformat=nolabel}

    \usepackage{adjustbox} % Used to constrain images to a maximum size 
    \usepackage{xcolor} % Allow colors to be defined
    \usepackage{enumerate} % Needed for markdown enumerations to work
    \usepackage{geometry} % Used to adjust the document margins
    \usepackage{amsmath} % Equations
    \usepackage{amssymb} % Equations
    \usepackage{textcomp} % defines textquotesingle
    % Hack from http://tex.stackexchange.com/a/47451/13684:
    \AtBeginDocument{%
        \def\PYZsq{\textquotesingle}% Upright quotes in Pygmentized code
    }
    \usepackage{upquote} % Upright quotes for verbatim code
    \usepackage{eurosym} % defines \euro
    \usepackage[mathletters]{ucs} % Extended unicode (utf-8) support
    \usepackage[utf8x]{inputenc} % Allow utf-8 characters in the tex document
    \usepackage{fancyvrb} % verbatim replacement that allows latex
    \usepackage{grffile} % extends the file name processing of package graphics 
                         % to support a larger range 
    % The hyperref package gives us a pdf with properly built
    % internal navigation ('pdf bookmarks' for the table of contents,
    % internal cross-reference links, web links for URLs, etc.)
    \usepackage{hyperref}
    \usepackage{longtable} % longtable support required by pandoc >1.10
    \usepackage{booktabs}  % table support for pandoc > 1.12.2
    \usepackage[inline]{enumitem} % IRkernel/repr support (it uses the enumerate* environment)
    \usepackage[normalem]{ulem} % ulem is needed to support strikethroughs (\sout)
                                % normalem makes italics be italics, not underlines
    

    
    
    % Colors for the hyperref package
    \definecolor{urlcolor}{rgb}{0,.145,.698}
    \definecolor{linkcolor}{rgb}{.71,0.21,0.01}
    \definecolor{citecolor}{rgb}{.12,.54,.11}

    % ANSI colors
    \definecolor{ansi-black}{HTML}{3E424D}
    \definecolor{ansi-black-intense}{HTML}{282C36}
    \definecolor{ansi-red}{HTML}{E75C58}
    \definecolor{ansi-red-intense}{HTML}{B22B31}
    \definecolor{ansi-green}{HTML}{00A250}
    \definecolor{ansi-green-intense}{HTML}{007427}
    \definecolor{ansi-yellow}{HTML}{DDB62B}
    \definecolor{ansi-yellow-intense}{HTML}{B27D12}
    \definecolor{ansi-blue}{HTML}{208FFB}
    \definecolor{ansi-blue-intense}{HTML}{0065CA}
    \definecolor{ansi-magenta}{HTML}{D160C4}
    \definecolor{ansi-magenta-intense}{HTML}{A03196}
    \definecolor{ansi-cyan}{HTML}{60C6C8}
    \definecolor{ansi-cyan-intense}{HTML}{258F8F}
    \definecolor{ansi-white}{HTML}{C5C1B4}
    \definecolor{ansi-white-intense}{HTML}{A1A6B2}

    % commands and environments needed by pandoc snippets
    % extracted from the output of `pandoc -s`
    \providecommand{\tightlist}{%
      \setlength{\itemsep}{0pt}\setlength{\parskip}{0pt}}
    \DefineVerbatimEnvironment{Highlighting}{Verbatim}{commandchars=\\\{\}}
    % Add ',fontsize=\small' for more characters per line
    \newenvironment{Shaded}{}{}
    \newcommand{\KeywordTok}[1]{\textcolor[rgb]{0.00,0.44,0.13}{\textbf{{#1}}}}
    \newcommand{\DataTypeTok}[1]{\textcolor[rgb]{0.56,0.13,0.00}{{#1}}}
    \newcommand{\DecValTok}[1]{\textcolor[rgb]{0.25,0.63,0.44}{{#1}}}
    \newcommand{\BaseNTok}[1]{\textcolor[rgb]{0.25,0.63,0.44}{{#1}}}
    \newcommand{\FloatTok}[1]{\textcolor[rgb]{0.25,0.63,0.44}{{#1}}}
    \newcommand{\CharTok}[1]{\textcolor[rgb]{0.25,0.44,0.63}{{#1}}}
    \newcommand{\StringTok}[1]{\textcolor[rgb]{0.25,0.44,0.63}{{#1}}}
    \newcommand{\CommentTok}[1]{\textcolor[rgb]{0.38,0.63,0.69}{\textit{{#1}}}}
    \newcommand{\OtherTok}[1]{\textcolor[rgb]{0.00,0.44,0.13}{{#1}}}
    \newcommand{\AlertTok}[1]{\textcolor[rgb]{1.00,0.00,0.00}{\textbf{{#1}}}}
    \newcommand{\FunctionTok}[1]{\textcolor[rgb]{0.02,0.16,0.49}{{#1}}}
    \newcommand{\RegionMarkerTok}[1]{{#1}}
    \newcommand{\ErrorTok}[1]{\textcolor[rgb]{1.00,0.00,0.00}{\textbf{{#1}}}}
    \newcommand{\NormalTok}[1]{{#1}}
    
    % Additional commands for more recent versions of Pandoc
    \newcommand{\ConstantTok}[1]{\textcolor[rgb]{0.53,0.00,0.00}{{#1}}}
    \newcommand{\SpecialCharTok}[1]{\textcolor[rgb]{0.25,0.44,0.63}{{#1}}}
    \newcommand{\VerbatimStringTok}[1]{\textcolor[rgb]{0.25,0.44,0.63}{{#1}}}
    \newcommand{\SpecialStringTok}[1]{\textcolor[rgb]{0.73,0.40,0.53}{{#1}}}
    \newcommand{\ImportTok}[1]{{#1}}
    \newcommand{\DocumentationTok}[1]{\textcolor[rgb]{0.73,0.13,0.13}{\textit{{#1}}}}
    \newcommand{\AnnotationTok}[1]{\textcolor[rgb]{0.38,0.63,0.69}{\textbf{\textit{{#1}}}}}
    \newcommand{\CommentVarTok}[1]{\textcolor[rgb]{0.38,0.63,0.69}{\textbf{\textit{{#1}}}}}
    \newcommand{\VariableTok}[1]{\textcolor[rgb]{0.10,0.09,0.49}{{#1}}}
    \newcommand{\ControlFlowTok}[1]{\textcolor[rgb]{0.00,0.44,0.13}{\textbf{{#1}}}}
    \newcommand{\OperatorTok}[1]{\textcolor[rgb]{0.40,0.40,0.40}{{#1}}}
    \newcommand{\BuiltInTok}[1]{{#1}}
    \newcommand{\ExtensionTok}[1]{{#1}}
    \newcommand{\PreprocessorTok}[1]{\textcolor[rgb]{0.74,0.48,0.00}{{#1}}}
    \newcommand{\AttributeTok}[1]{\textcolor[rgb]{0.49,0.56,0.16}{{#1}}}
    \newcommand{\InformationTok}[1]{\textcolor[rgb]{0.38,0.63,0.69}{\textbf{\textit{{#1}}}}}
    \newcommand{\WarningTok}[1]{\textcolor[rgb]{0.38,0.63,0.69}{\textbf{\textit{{#1}}}}}
    
    
    % Define a nice break command that doesn't care if a line doesn't already
    % exist.
    \def\br{\hspace*{\fill} \\* }
    % Math Jax compatability definitions
    \def\gt{>}
    \def\lt{<}
    % Document parameters
    \title{pw}
    
    
    

    % Pygments definitions
    
\makeatletter
\def\PY@reset{\let\PY@it=\relax \let\PY@bf=\relax%
    \let\PY@ul=\relax \let\PY@tc=\relax%
    \let\PY@bc=\relax \let\PY@ff=\relax}
\def\PY@tok#1{\csname PY@tok@#1\endcsname}
\def\PY@toks#1+{\ifx\relax#1\empty\else%
    \PY@tok{#1}\expandafter\PY@toks\fi}
\def\PY@do#1{\PY@bc{\PY@tc{\PY@ul{%
    \PY@it{\PY@bf{\PY@ff{#1}}}}}}}
\def\PY#1#2{\PY@reset\PY@toks#1+\relax+\PY@do{#2}}

\expandafter\def\csname PY@tok@w\endcsname{\def\PY@tc##1{\textcolor[rgb]{0.73,0.73,0.73}{##1}}}
\expandafter\def\csname PY@tok@c\endcsname{\let\PY@it=\textit\def\PY@tc##1{\textcolor[rgb]{0.25,0.50,0.50}{##1}}}
\expandafter\def\csname PY@tok@cp\endcsname{\def\PY@tc##1{\textcolor[rgb]{0.74,0.48,0.00}{##1}}}
\expandafter\def\csname PY@tok@k\endcsname{\let\PY@bf=\textbf\def\PY@tc##1{\textcolor[rgb]{0.00,0.50,0.00}{##1}}}
\expandafter\def\csname PY@tok@kp\endcsname{\def\PY@tc##1{\textcolor[rgb]{0.00,0.50,0.00}{##1}}}
\expandafter\def\csname PY@tok@kt\endcsname{\def\PY@tc##1{\textcolor[rgb]{0.69,0.00,0.25}{##1}}}
\expandafter\def\csname PY@tok@o\endcsname{\def\PY@tc##1{\textcolor[rgb]{0.40,0.40,0.40}{##1}}}
\expandafter\def\csname PY@tok@ow\endcsname{\let\PY@bf=\textbf\def\PY@tc##1{\textcolor[rgb]{0.67,0.13,1.00}{##1}}}
\expandafter\def\csname PY@tok@nb\endcsname{\def\PY@tc##1{\textcolor[rgb]{0.00,0.50,0.00}{##1}}}
\expandafter\def\csname PY@tok@nf\endcsname{\def\PY@tc##1{\textcolor[rgb]{0.00,0.00,1.00}{##1}}}
\expandafter\def\csname PY@tok@nc\endcsname{\let\PY@bf=\textbf\def\PY@tc##1{\textcolor[rgb]{0.00,0.00,1.00}{##1}}}
\expandafter\def\csname PY@tok@nn\endcsname{\let\PY@bf=\textbf\def\PY@tc##1{\textcolor[rgb]{0.00,0.00,1.00}{##1}}}
\expandafter\def\csname PY@tok@ne\endcsname{\let\PY@bf=\textbf\def\PY@tc##1{\textcolor[rgb]{0.82,0.25,0.23}{##1}}}
\expandafter\def\csname PY@tok@nv\endcsname{\def\PY@tc##1{\textcolor[rgb]{0.10,0.09,0.49}{##1}}}
\expandafter\def\csname PY@tok@no\endcsname{\def\PY@tc##1{\textcolor[rgb]{0.53,0.00,0.00}{##1}}}
\expandafter\def\csname PY@tok@nl\endcsname{\def\PY@tc##1{\textcolor[rgb]{0.63,0.63,0.00}{##1}}}
\expandafter\def\csname PY@tok@ni\endcsname{\let\PY@bf=\textbf\def\PY@tc##1{\textcolor[rgb]{0.60,0.60,0.60}{##1}}}
\expandafter\def\csname PY@tok@na\endcsname{\def\PY@tc##1{\textcolor[rgb]{0.49,0.56,0.16}{##1}}}
\expandafter\def\csname PY@tok@nt\endcsname{\let\PY@bf=\textbf\def\PY@tc##1{\textcolor[rgb]{0.00,0.50,0.00}{##1}}}
\expandafter\def\csname PY@tok@nd\endcsname{\def\PY@tc##1{\textcolor[rgb]{0.67,0.13,1.00}{##1}}}
\expandafter\def\csname PY@tok@s\endcsname{\def\PY@tc##1{\textcolor[rgb]{0.73,0.13,0.13}{##1}}}
\expandafter\def\csname PY@tok@sd\endcsname{\let\PY@it=\textit\def\PY@tc##1{\textcolor[rgb]{0.73,0.13,0.13}{##1}}}
\expandafter\def\csname PY@tok@si\endcsname{\let\PY@bf=\textbf\def\PY@tc##1{\textcolor[rgb]{0.73,0.40,0.53}{##1}}}
\expandafter\def\csname PY@tok@se\endcsname{\let\PY@bf=\textbf\def\PY@tc##1{\textcolor[rgb]{0.73,0.40,0.13}{##1}}}
\expandafter\def\csname PY@tok@sr\endcsname{\def\PY@tc##1{\textcolor[rgb]{0.73,0.40,0.53}{##1}}}
\expandafter\def\csname PY@tok@ss\endcsname{\def\PY@tc##1{\textcolor[rgb]{0.10,0.09,0.49}{##1}}}
\expandafter\def\csname PY@tok@sx\endcsname{\def\PY@tc##1{\textcolor[rgb]{0.00,0.50,0.00}{##1}}}
\expandafter\def\csname PY@tok@m\endcsname{\def\PY@tc##1{\textcolor[rgb]{0.40,0.40,0.40}{##1}}}
\expandafter\def\csname PY@tok@gh\endcsname{\let\PY@bf=\textbf\def\PY@tc##1{\textcolor[rgb]{0.00,0.00,0.50}{##1}}}
\expandafter\def\csname PY@tok@gu\endcsname{\let\PY@bf=\textbf\def\PY@tc##1{\textcolor[rgb]{0.50,0.00,0.50}{##1}}}
\expandafter\def\csname PY@tok@gd\endcsname{\def\PY@tc##1{\textcolor[rgb]{0.63,0.00,0.00}{##1}}}
\expandafter\def\csname PY@tok@gi\endcsname{\def\PY@tc##1{\textcolor[rgb]{0.00,0.63,0.00}{##1}}}
\expandafter\def\csname PY@tok@gr\endcsname{\def\PY@tc##1{\textcolor[rgb]{1.00,0.00,0.00}{##1}}}
\expandafter\def\csname PY@tok@ge\endcsname{\let\PY@it=\textit}
\expandafter\def\csname PY@tok@gs\endcsname{\let\PY@bf=\textbf}
\expandafter\def\csname PY@tok@gp\endcsname{\let\PY@bf=\textbf\def\PY@tc##1{\textcolor[rgb]{0.00,0.00,0.50}{##1}}}
\expandafter\def\csname PY@tok@go\endcsname{\def\PY@tc##1{\textcolor[rgb]{0.53,0.53,0.53}{##1}}}
\expandafter\def\csname PY@tok@gt\endcsname{\def\PY@tc##1{\textcolor[rgb]{0.00,0.27,0.87}{##1}}}
\expandafter\def\csname PY@tok@err\endcsname{\def\PY@bc##1{\setlength{\fboxsep}{0pt}\fcolorbox[rgb]{1.00,0.00,0.00}{1,1,1}{\strut ##1}}}
\expandafter\def\csname PY@tok@kc\endcsname{\let\PY@bf=\textbf\def\PY@tc##1{\textcolor[rgb]{0.00,0.50,0.00}{##1}}}
\expandafter\def\csname PY@tok@kd\endcsname{\let\PY@bf=\textbf\def\PY@tc##1{\textcolor[rgb]{0.00,0.50,0.00}{##1}}}
\expandafter\def\csname PY@tok@kn\endcsname{\let\PY@bf=\textbf\def\PY@tc##1{\textcolor[rgb]{0.00,0.50,0.00}{##1}}}
\expandafter\def\csname PY@tok@kr\endcsname{\let\PY@bf=\textbf\def\PY@tc##1{\textcolor[rgb]{0.00,0.50,0.00}{##1}}}
\expandafter\def\csname PY@tok@bp\endcsname{\def\PY@tc##1{\textcolor[rgb]{0.00,0.50,0.00}{##1}}}
\expandafter\def\csname PY@tok@fm\endcsname{\def\PY@tc##1{\textcolor[rgb]{0.00,0.00,1.00}{##1}}}
\expandafter\def\csname PY@tok@vc\endcsname{\def\PY@tc##1{\textcolor[rgb]{0.10,0.09,0.49}{##1}}}
\expandafter\def\csname PY@tok@vg\endcsname{\def\PY@tc##1{\textcolor[rgb]{0.10,0.09,0.49}{##1}}}
\expandafter\def\csname PY@tok@vi\endcsname{\def\PY@tc##1{\textcolor[rgb]{0.10,0.09,0.49}{##1}}}
\expandafter\def\csname PY@tok@vm\endcsname{\def\PY@tc##1{\textcolor[rgb]{0.10,0.09,0.49}{##1}}}
\expandafter\def\csname PY@tok@sa\endcsname{\def\PY@tc##1{\textcolor[rgb]{0.73,0.13,0.13}{##1}}}
\expandafter\def\csname PY@tok@sb\endcsname{\def\PY@tc##1{\textcolor[rgb]{0.73,0.13,0.13}{##1}}}
\expandafter\def\csname PY@tok@sc\endcsname{\def\PY@tc##1{\textcolor[rgb]{0.73,0.13,0.13}{##1}}}
\expandafter\def\csname PY@tok@dl\endcsname{\def\PY@tc##1{\textcolor[rgb]{0.73,0.13,0.13}{##1}}}
\expandafter\def\csname PY@tok@s2\endcsname{\def\PY@tc##1{\textcolor[rgb]{0.73,0.13,0.13}{##1}}}
\expandafter\def\csname PY@tok@sh\endcsname{\def\PY@tc##1{\textcolor[rgb]{0.73,0.13,0.13}{##1}}}
\expandafter\def\csname PY@tok@s1\endcsname{\def\PY@tc##1{\textcolor[rgb]{0.73,0.13,0.13}{##1}}}
\expandafter\def\csname PY@tok@mb\endcsname{\def\PY@tc##1{\textcolor[rgb]{0.40,0.40,0.40}{##1}}}
\expandafter\def\csname PY@tok@mf\endcsname{\def\PY@tc##1{\textcolor[rgb]{0.40,0.40,0.40}{##1}}}
\expandafter\def\csname PY@tok@mh\endcsname{\def\PY@tc##1{\textcolor[rgb]{0.40,0.40,0.40}{##1}}}
\expandafter\def\csname PY@tok@mi\endcsname{\def\PY@tc##1{\textcolor[rgb]{0.40,0.40,0.40}{##1}}}
\expandafter\def\csname PY@tok@il\endcsname{\def\PY@tc##1{\textcolor[rgb]{0.40,0.40,0.40}{##1}}}
\expandafter\def\csname PY@tok@mo\endcsname{\def\PY@tc##1{\textcolor[rgb]{0.40,0.40,0.40}{##1}}}
\expandafter\def\csname PY@tok@ch\endcsname{\let\PY@it=\textit\def\PY@tc##1{\textcolor[rgb]{0.25,0.50,0.50}{##1}}}
\expandafter\def\csname PY@tok@cm\endcsname{\let\PY@it=\textit\def\PY@tc##1{\textcolor[rgb]{0.25,0.50,0.50}{##1}}}
\expandafter\def\csname PY@tok@cpf\endcsname{\let\PY@it=\textit\def\PY@tc##1{\textcolor[rgb]{0.25,0.50,0.50}{##1}}}
\expandafter\def\csname PY@tok@c1\endcsname{\let\PY@it=\textit\def\PY@tc##1{\textcolor[rgb]{0.25,0.50,0.50}{##1}}}
\expandafter\def\csname PY@tok@cs\endcsname{\let\PY@it=\textit\def\PY@tc##1{\textcolor[rgb]{0.25,0.50,0.50}{##1}}}

\def\PYZbs{\char`\\}
\def\PYZus{\char`\_}
\def\PYZob{\char`\{}
\def\PYZcb{\char`\}}
\def\PYZca{\char`\^}
\def\PYZam{\char`\&}
\def\PYZlt{\char`\<}
\def\PYZgt{\char`\>}
\def\PYZsh{\char`\#}
\def\PYZpc{\char`\%}
\def\PYZdl{\char`\$}
\def\PYZhy{\char`\-}
\def\PYZsq{\char`\'}
\def\PYZdq{\char`\"}
\def\PYZti{\char`\~}
% for compatibility with earlier versions
\def\PYZat{@}
\def\PYZlb{[}
\def\PYZrb{]}
\makeatother


    % Exact colors from NB
    \definecolor{incolor}{rgb}{0.0, 0.0, 0.5}
    \definecolor{outcolor}{rgb}{0.545, 0.0, 0.0}



    
    % Prevent overflowing lines due to hard-to-break entities
    \sloppy 
    % Setup hyperref package
    \hypersetup{
      breaklinks=true,  % so long urls are correctly broken across lines
      colorlinks=true,
      urlcolor=urlcolor,
      linkcolor=linkcolor,
      citecolor=citecolor,
      }
    % Slightly bigger margins than the latex defaults
    
    \geometry{verbose,tmargin=1in,bmargin=1in,lmargin=1in,rmargin=1in}
    
    

    \begin{document}
    
    
    \maketitle
    
    

    
    \begin{Verbatim}[commandchars=\\\{\}]
{\color{incolor}In [{\color{incolor}4}]:} \PY{o}{\PYZpc{}}\PY{k}{logstop}
        \PY{o}{\PYZpc{}}\PY{k}{logstart} \PYZhy{}rtq \PYZti{}/.logs/pw.py append
        \PY{o}{\PYZpc{}}\PY{k}{matplotlib} inline
        \PY{k+kn}{import} \PY{n+nn}{matplotlib}
        \PY{k+kn}{import} \PY{n+nn}{seaborn} \PY{k}{as} \PY{n+nn}{sns}
        \PY{n}{sns}\PY{o}{.}\PY{n}{set}\PY{p}{(}\PY{p}{)}
        \PY{n}{matplotlib}\PY{o}{.}\PY{n}{rcParams}\PY{p}{[}\PY{l+s+s1}{\PYZsq{}}\PY{l+s+s1}{figure.dpi}\PY{l+s+s1}{\PYZsq{}}\PY{p}{]} \PY{o}{=} \PY{l+m+mi}{144}
\end{Verbatim}


    \begin{Verbatim}[commandchars=\\\{\}]
{\color{incolor}In [{\color{incolor}3}]:} \PY{k+kn}{from} \PY{n+nn}{static\PYZus{}grader} \PY{k+kn}{import} \PY{n}{grader}
\end{Verbatim}


    \hypertarget{pw-miniproject}{%
\section{PW Miniproject}\label{pw-miniproject}}

\hypertarget{introduction}{%
\subsection{Introduction}\label{introduction}}

The objective of this miniproject is to exercise your ability to use
basic Python data structures, define functions, and control program
flow. We will be using these concepts to perform some fundamental data
wrangling tasks such as joining data sets together, splitting data into
groups, and aggregating data into summary statistics. \textbf{Please do
not use \texttt{pandas} or \texttt{numpy} to answer these questions.}

We will be working with medical data from the British NHS on
prescription drugs. Since this is real data, it contains many
ambiguities that we will need to confront in our analysis. This is
commonplace in data science, and is one of the lessons you will learn in
this miniproject.

    \hypertarget{downloading-the-data}{%
\subsection{Downloading the data}\label{downloading-the-data}}

We first need to download the data we'll be using from Amazon S3:

    \begin{Verbatim}[commandchars=\\\{\}]
{\color{incolor}In [{\color{incolor}5}]:} \PYZpc{}\PYZpc{}bash
        mkdir pw\PYZhy{}data
        wget http://dataincubator\PYZhy{}wqu.s3.amazonaws.com/pwdata/201701scripts\PYZus{}sample.json.gz \PYZhy{}nc \PYZhy{}P ./pw\PYZhy{}data
        wget http://dataincubator\PYZhy{}wqu.s3.amazonaws.com/pwdata/practices.json.gz \PYZhy{}nc \PYZhy{}P ./pw\PYZhy{}data
\end{Verbatim}


    \begin{Verbatim}[commandchars=\\\{\}]
mkdir: cannot create directory ‘pw-data’: File exists
File ‘./pw-data/201701scripts\_sample.json.gz’ already there; not retrieving.

File ‘./pw-data/practices.json.gz’ already there; not retrieving.


    \end{Verbatim}

    \hypertarget{loading-the-data}{%
\subsection{Loading the data}\label{loading-the-data}}

The first step of the project is to read in the data. We will discuss
reading and writing various kinds of files later in the course, but the
code below should get you started.

    \begin{Verbatim}[commandchars=\\\{\}]
{\color{incolor}In [{\color{incolor}6}]:} \PY{k+kn}{import} \PY{n+nn}{gzip}
        \PY{k+kn}{import} \PY{n+nn}{simplejson} \PY{k}{as} \PY{n+nn}{json}
\end{Verbatim}


    \begin{Verbatim}[commandchars=\\\{\}]
{\color{incolor}In [{\color{incolor}7}]:} \PY{k}{with} \PY{n}{gzip}\PY{o}{.}\PY{n}{open}\PY{p}{(}\PY{l+s+s1}{\PYZsq{}}\PY{l+s+s1}{./pw\PYZhy{}data/201701scripts\PYZus{}sample.json.gz}\PY{l+s+s1}{\PYZsq{}}\PY{p}{,} \PY{l+s+s1}{\PYZsq{}}\PY{l+s+s1}{rb}\PY{l+s+s1}{\PYZsq{}}\PY{p}{)} \PY{k}{as} \PY{n}{f}\PY{p}{:}
            \PY{n}{scripts} \PY{o}{=} \PY{n}{json}\PY{o}{.}\PY{n}{load}\PY{p}{(}\PY{n}{f}\PY{p}{)}
        
        \PY{k}{with} \PY{n}{gzip}\PY{o}{.}\PY{n}{open}\PY{p}{(}\PY{l+s+s1}{\PYZsq{}}\PY{l+s+s1}{./pw\PYZhy{}data/practices.json.gz}\PY{l+s+s1}{\PYZsq{}}\PY{p}{,} \PY{l+s+s1}{\PYZsq{}}\PY{l+s+s1}{rb}\PY{l+s+s1}{\PYZsq{}}\PY{p}{)} \PY{k}{as} \PY{n}{f}\PY{p}{:}
            \PY{n}{practices} \PY{o}{=} \PY{n}{json}\PY{o}{.}\PY{n}{load}\PY{p}{(}\PY{n}{f}\PY{p}{)}
\end{Verbatim}


    This data set comes from Britain's National Health Service. The
\texttt{scripts} variable is a list of prescriptions issued by NHS
doctors. Each prescription is represented by a dictionary with various
data fields: \texttt{\textquotesingle{}practice\textquotesingle{}},
\texttt{\textquotesingle{}bnf\_code\textquotesingle{}},
\texttt{\textquotesingle{}bnf\_name\textquotesingle{}},
\texttt{\textquotesingle{}quantity\textquotesingle{}},
\texttt{\textquotesingle{}items\textquotesingle{}},
\texttt{\textquotesingle{}nic\textquotesingle{}}, and
\texttt{\textquotesingle{}act\_cost\textquotesingle{}}.

    \begin{Verbatim}[commandchars=\\\{\}]
{\color{incolor}In [{\color{incolor}5}]:} \PY{n}{scripts}\PY{p}{[}\PY{p}{:}\PY{l+m+mi}{2}\PY{p}{]}
\end{Verbatim}


\begin{Verbatim}[commandchars=\\\{\}]
{\color{outcolor}Out[{\color{outcolor}5}]:} [\{'bnf\_code': '0101010G0AAABAB',
          'items': 2,
          'practice': 'N81013',
          'bnf\_name': 'Co-Magaldrox\_Susp 195mg/220mg/5ml S/F',
          'nic': 5.98,
          'act\_cost': 5.56,
          'quantity': 1000\},
         \{'bnf\_code': '0101021B0AAAHAH',
          'items': 1,
          'practice': 'N81013',
          'bnf\_name': 'Alginate\_Raft-Forming Oral Susp S/F',
          'nic': 1.95,
          'act\_cost': 1.82,
          'quantity': 500\}]
\end{Verbatim}
            
    A
\href{http://webarchive.nationalarchives.gov.uk/20180328130852tf_/http://content.digital.nhs.uk/media/10686/Download-glossary-of-terms-for-GP-prescribing---presentation-level/pdf/PLP_Presentation_Level_Glossary_April_2015.pdf/}{glossary
of terms} and
\href{http://webarchive.nationalarchives.gov.uk/20180328130852tf_/http://content.digital.nhs.uk/media/10048/FAQs-Practice-Level-Prescribingpdf/pdf/PLP_FAQs_April_2015.pdf/}{FAQ}
is available from the NHS regarding the data. Below we supply a data
dictionary briefly describing what these fields mean.

\begin{longtable}[]{@{}cl@{}}
\toprule
\begin{minipage}[b]{0.49\columnwidth}\centering
Data field\strut
\end{minipage} & \begin{minipage}[b]{0.45\columnwidth}\raggedright
Description\strut
\end{minipage}\tabularnewline
\midrule
\endhead
\begin{minipage}[t]{0.49\columnwidth}\centering
\texttt{\textquotesingle{}practice\textquotesingle{}}\strut
\end{minipage} & \begin{minipage}[t]{0.45\columnwidth}\raggedright
Code designating the medical practice issuing the prescription\strut
\end{minipage}\tabularnewline
\begin{minipage}[t]{0.49\columnwidth}\centering
\texttt{\textquotesingle{}bnf\_code\textquotesingle{}}\strut
\end{minipage} & \begin{minipage}[t]{0.45\columnwidth}\raggedright
British National Formulary drug code\strut
\end{minipage}\tabularnewline
\begin{minipage}[t]{0.49\columnwidth}\centering
\texttt{\textquotesingle{}bnf\_name\textquotesingle{}}\strut
\end{minipage} & \begin{minipage}[t]{0.45\columnwidth}\raggedright
British National Formulary drug name\strut
\end{minipage}\tabularnewline
\begin{minipage}[t]{0.49\columnwidth}\centering
\texttt{\textquotesingle{}quantity\textquotesingle{}}\strut
\end{minipage} & \begin{minipage}[t]{0.45\columnwidth}\raggedright
Number of capsules/quantity of liquid/grams of powder prescribed\strut
\end{minipage}\tabularnewline
\begin{minipage}[t]{0.49\columnwidth}\centering
\texttt{\textquotesingle{}items\textquotesingle{}}\strut
\end{minipage} & \begin{minipage}[t]{0.45\columnwidth}\raggedright
Number of refills (e.g.~if
\texttt{\textquotesingle{}quantity\textquotesingle{}} is 30 capsules, 3
\texttt{\textquotesingle{}items\textquotesingle{}} means 3 bottles of 30
capsules)\strut
\end{minipage}\tabularnewline
\begin{minipage}[t]{0.49\columnwidth}\centering
\texttt{\textquotesingle{}nic\textquotesingle{}}\strut
\end{minipage} & \begin{minipage}[t]{0.45\columnwidth}\raggedright
Net ingredient cost\strut
\end{minipage}\tabularnewline
\begin{minipage}[t]{0.49\columnwidth}\centering
\texttt{\textquotesingle{}act\_cost\textquotesingle{}}\strut
\end{minipage} & \begin{minipage}[t]{0.45\columnwidth}\raggedright
Total cost including containers, fees, and discounts\strut
\end{minipage}\tabularnewline
\bottomrule
\end{longtable}

    The \texttt{practices} variable is a list of member medical practices of
the NHS. Each practice is represented by a dictionary containing
identifying information for the medical practice. Most of the data
fields are self-explanatory. Notice the values in the
\texttt{\textquotesingle{}code\textquotesingle{}} field of
\texttt{practices} match the values in the
\texttt{\textquotesingle{}practice\textquotesingle{}} field of
\texttt{scripts}.

    \begin{Verbatim}[commandchars=\\\{\}]
{\color{incolor}In [{\color{incolor}7}]:} \PY{n}{practices}\PY{p}{[}\PY{p}{:}\PY{l+m+mi}{2}\PY{p}{]}
\end{Verbatim}


\begin{Verbatim}[commandchars=\\\{\}]
{\color{outcolor}Out[{\color{outcolor}7}]:} [\{'code': 'A81001',
          'name': 'THE DENSHAM SURGERY',
          'addr\_1': 'THE HEALTH CENTRE',
          'addr\_2': 'LAWSON STREET',
          'borough': 'STOCKTON ON TEES',
          'village': 'CLEVELAND',
          'post\_code': 'TS18 1HU'\},
         \{'code': 'A81002',
          'name': 'QUEENS PARK MEDICAL CENTRE',
          'addr\_1': 'QUEENS PARK MEDICAL CTR',
          'addr\_2': 'FARRER STREET',
          'borough': 'STOCKTON ON TEES',
          'village': 'CLEVELAND',
          'post\_code': 'TS18 2AW'\}]
\end{Verbatim}
            
    In the following questions we will ask you to explore this data set. You
may need to combine pieces of the data set together in order to answer
some questions. Not every element of the data set will be used in
answering the questions.

    \hypertarget{question-1-summary_statistics}{%
\subsection{Question 1:
summary\_statistics}\label{question-1-summary_statistics}}

Our beneficiary data (\texttt{scripts}) contains quantitative data on
the number of items dispensed
(\texttt{\textquotesingle{}items\textquotesingle{}}), the total quantity
of item dispensed
(\texttt{\textquotesingle{}quantity\textquotesingle{}}), the net cost of
the ingredients (\texttt{\textquotesingle{}nic\textquotesingle{}}), and
the actual cost to the patient
(\texttt{\textquotesingle{}act\_cost\textquotesingle{}}). Whenever
working with a new data set, it can be useful to calculate summary
statistics to develop a feeling for the volume and character of the
data. This makes it easier to spot trends and significant features
during further stages of analysis.

Calculate the sum, mean, standard deviation, and quartile statistics for
each of these quantities. Format your results for each quantity as a
list:
\texttt{{[}sum,\ mean,\ standard\ deviation,\ 1st\ quartile,\ median,\ 3rd\ quartile{]}}.
We'll create a \texttt{tuple} with these lists for each quantity as a
final result.

    \begin{Verbatim}[commandchars=\\\{\}]
{\color{incolor}In [{\color{incolor}8}]:} \PY{k}{def} \PY{n+nf}{describe}\PY{p}{(}\PY{n}{key}\PY{p}{)}\PY{p}{:}
            
            \PY{k}{try}\PY{p}{:} 
                \PY{n}{total} \PY{o}{=} \PY{n+nb}{sum} \PY{p}{(}\PY{n}{scripts}\PY{p}{[}\PY{n}{i}\PY{p}{]}\PY{p}{[}\PY{n}{key}\PY{p}{]} \PY{k}{for} \PY{n}{i} \PY{o+ow}{in} \PY{n+nb}{range}\PY{p}{(}\PY{n+nb}{len}\PY{p}{(}\PY{n}{scripts}\PY{p}{)}\PY{p}{)}\PY{p}{)}
                \PY{n}{avg} \PY{o}{=} \PY{n}{total}\PY{o}{/}\PY{n+nb}{len}\PY{p}{(}\PY{n}{scripts}\PY{p}{)}
                \PY{n}{s} \PY{o}{=}  \PY{p}{(}\PY{n+nb}{sum}\PY{p}{(}\PY{p}{(}\PY{n+nb}{float}\PY{p}{(}\PY{n}{scripts}\PY{p}{[}\PY{n}{i}\PY{p}{]}\PY{p}{[}\PY{n}{key}\PY{p}{]} \PY{o}{\PYZhy{}} \PY{n}{avg}\PY{p}{)}\PY{o}{*}\PY{o}{*}\PY{l+m+mi}{2} \PY{k}{for} \PY{n}{i} \PY{o+ow}{in} \PY{n+nb}{range}\PY{p}{(}\PY{l+m+mi}{0}\PY{p}{,}\PY{n+nb}{len}\PY{p}{(}\PY{n}{scripts}\PY{p}{)}\PY{p}{)}\PY{p}{)}\PY{p}{)}\PY{o}{/}\PY{p}{(}\PY{n+nb}{len}\PY{p}{(}\PY{n}{scripts}\PY{p}{)} \PY{o}{\PYZhy{}} \PY{l+m+mi}{1}\PY{p}{)}\PY{p}{)}\PY{o}{*}\PY{o}{*}\PY{l+m+mf}{0.5}
                \PY{n}{sorted\PYZus{}key} \PY{o}{=} \PY{n+nb}{sorted}\PY{p}{(}\PY{n}{scripts}\PY{p}{,} \PY{n}{key} \PY{o}{=} \PY{k}{lambda} \PY{n}{i}\PY{p}{:} \PY{n}{i}\PY{p}{[}\PY{n}{key}\PY{p}{]}\PY{p}{)}
                \PY{c+c1}{\PYZsh{}sorted\PYZus{}key = scripts[i][key].sort()}
                \PY{n}{q25} \PY{o}{=} \PY{n}{sorted\PYZus{}key}\PY{p}{[}\PY{n+nb}{int}\PY{p}{(}\PY{n+nb}{len}\PY{p}{(}\PY{n}{scripts}\PY{p}{)}\PY{o}{/}\PY{l+m+mi}{4}\PY{p}{)}\PY{p}{]}\PY{p}{[}\PY{n}{key}\PY{p}{]}
                \PY{n}{med} \PY{o}{=} \PY{n}{sorted\PYZus{}key}\PY{p}{[}\PY{n+nb}{int}\PY{p}{(}\PY{n+nb}{len}\PY{p}{(}\PY{n}{scripts}\PY{p}{)}\PY{o}{/}\PY{l+m+mi}{2}\PY{p}{)}\PY{p}{]}\PY{p}{[}\PY{n}{key}\PY{p}{]}
                \PY{n}{q75} \PY{o}{=} \PY{n}{sorted\PYZus{}key}\PY{p}{[}\PY{n+nb}{int}\PY{p}{(}\PY{n+nb}{len}\PY{p}{(}\PY{n}{scripts}\PY{p}{)}\PY{o}{*}\PY{l+m+mi}{3}\PY{o}{/}\PY{l+m+mi}{4}\PY{p}{)}\PY{p}{]}\PY{p}{[}\PY{n}{key}\PY{p}{]}
            \PY{k}{except} \PY{n+ne}{ValueError}\PY{p}{:} 
                \PY{n+nb}{print}\PY{p}{(}\PY{l+s+s2}{\PYZdq{}}\PY{l+s+s2}{oops}\PY{l+s+s2}{\PYZdq{}}\PY{p}{)}
               
                
            \PY{k}{return} \PY{p}{(}\PY{n}{total}\PY{p}{,} \PY{n}{avg}\PY{p}{,} \PY{n}{s}\PY{p}{,} \PY{n}{q25}\PY{p}{,} \PY{n}{med}\PY{p}{,} \PY{n}{q75}\PY{p}{)}
\end{Verbatim}


    \begin{Verbatim}[commandchars=\\\{\}]
{\color{incolor}In [{\color{incolor}9}]:} \PY{n}{summary} \PY{o}{=} \PY{p}{[}\PY{p}{(}\PY{l+s+s1}{\PYZsq{}}\PY{l+s+s1}{items}\PY{l+s+s1}{\PYZsq{}}\PY{p}{,} \PY{n}{describe}\PY{p}{(}\PY{l+s+s1}{\PYZsq{}}\PY{l+s+s1}{items}\PY{l+s+s1}{\PYZsq{}}\PY{p}{)}\PY{p}{)}\PY{p}{,}
                   \PY{p}{(}\PY{l+s+s1}{\PYZsq{}}\PY{l+s+s1}{quantity}\PY{l+s+s1}{\PYZsq{}}\PY{p}{,} \PY{n}{describe}\PY{p}{(}\PY{l+s+s1}{\PYZsq{}}\PY{l+s+s1}{quantity}\PY{l+s+s1}{\PYZsq{}}\PY{p}{)}\PY{p}{)}\PY{p}{,}
                   \PY{p}{(}\PY{l+s+s1}{\PYZsq{}}\PY{l+s+s1}{nic}\PY{l+s+s1}{\PYZsq{}}\PY{p}{,} \PY{n}{describe}\PY{p}{(}\PY{l+s+s1}{\PYZsq{}}\PY{l+s+s1}{nic}\PY{l+s+s1}{\PYZsq{}}\PY{p}{)}\PY{p}{)}\PY{p}{,}
                   \PY{p}{(}\PY{l+s+s1}{\PYZsq{}}\PY{l+s+s1}{act\PYZus{}cost}\PY{l+s+s1}{\PYZsq{}}\PY{p}{,} \PY{n}{describe}\PY{p}{(}\PY{l+s+s1}{\PYZsq{}}\PY{l+s+s1}{act\PYZus{}cost}\PY{l+s+s1}{\PYZsq{}}\PY{p}{)}\PY{p}{)}\PY{p}{]}
        \PY{n+nb}{print}\PY{p}{(}\PY{n}{summary}\PY{p}{)}
\end{Verbatim}


    \begin{Verbatim}[commandchars=\\\{\}]
[('items', (4410054, 11.522744731217633, 33.11220959820685, 1, 3, 8)), ('quantity', (316356836, 826.5883059943667, 3872.186073305146, 30, 120, 466)), ('nic', (29048309.790000338, 75.89844899484315, 197.57308474088356, 7.7, 22.62, 65.94)), ('act\_cost', (27053937.599999707, 70.68748295124895, 183.26755837716715, 7.25, 21.24, 61.53))]

    \end{Verbatim}

    \begin{Verbatim}[commandchars=\\\{\}]
{\color{incolor}In [{\color{incolor}10}]:} \PY{n}{grader}\PY{o}{.}\PY{n}{score}\PY{o}{.}\PY{n}{pw\PYZus{}\PYZus{}summary\PYZus{}statistics}\PY{p}{(}\PY{n}{summary}\PY{p}{)}
\end{Verbatim}


    \begin{Verbatim}[commandchars=\\\{\}]
==================
Your score:  1.0
==================

    \end{Verbatim}

    \hypertarget{question-2-most_common_item}{%
\subsection{Question 2:
most\_common\_item}\label{question-2-most_common_item}}

Often we are not interested only in how the data is distributed in our
entire data set, but within particular groups -- for example, how many
items of each drug (i.e.
\texttt{\textquotesingle{}bnf\_name\textquotesingle{}}) were prescribed?
Calculate the total items prescribed for each
\texttt{\textquotesingle{}bnf\_name\textquotesingle{}}. What is the most
commonly prescribed
\texttt{\textquotesingle{}bnf\_name\textquotesingle{}} in our data?

To calculate this, we first need to split our data set into groups
corresponding with the different values of
\texttt{\textquotesingle{}bnf\_name\textquotesingle{}}. Then we can sum
the number of items dispensed within in each group. Finally we can find
the largest sum.

We'll use \texttt{\textquotesingle{}bnf\_name\textquotesingle{}} to
construct our groups. You should have \emph{5619} unique values for
\texttt{\textquotesingle{}bnf\_name\textquotesingle{}}.

    \begin{Verbatim}[commandchars=\\\{\}]
{\color{incolor}In [{\color{incolor}11}]:} \PY{n}{bnf\PYZus{}names} \PY{o}{=} \PY{p}{[}\PY{n}{scripts}\PY{p}{[}\PY{n}{key}\PY{p}{]}\PY{p}{[}\PY{l+s+s1}{\PYZsq{}}\PY{l+s+s1}{bnf\PYZus{}name}\PY{l+s+s1}{\PYZsq{}}\PY{p}{]} \PY{k}{for} \PY{n}{key} \PY{o+ow}{in} \PY{n+nb}{range}\PY{p}{(}\PY{n+nb}{len}\PY{p}{(}\PY{n}{scripts}\PY{p}{)}\PY{p}{)}\PY{p}{]}
\end{Verbatim}


    \begin{Verbatim}[commandchars=\\\{\}]
{\color{incolor}In [{\color{incolor}12}]:} \PY{n}{scripts}\PY{p}{[}\PY{l+m+mi}{1}\PY{p}{]}\PY{p}{[}\PY{l+s+s1}{\PYZsq{}}\PY{l+s+s1}{bnf\PYZus{}name}\PY{l+s+s1}{\PYZsq{}}\PY{p}{]}
         \PY{n}{bnf\PYZus{}names}\PY{p}{[}\PY{l+m+mi}{0}\PY{p}{:}\PY{l+m+mi}{5}\PY{p}{]}
\end{Verbatim}


\begin{Verbatim}[commandchars=\\\{\}]
{\color{outcolor}Out[{\color{outcolor}12}]:} ['Co-Magaldrox\_Susp 195mg/220mg/5ml S/F',
          'Alginate\_Raft-Forming Oral Susp S/F',
          'Sod Algin/Pot Bicarb\_Susp S/F',
          'Sod Alginate/Pot Bicarb\_Tab Chble 500mg',
          'Gaviscon Infant\_Sach 2g (Dual Pack) S/F']
\end{Verbatim}
            
    We want to construct ``groups'' identified by
\texttt{\textquotesingle{}bnf\_name\textquotesingle{}}, where each group
is a collection of prescriptions (i.e.~dictionaries from
\texttt{scripts}). We'll construct a dictionary called \texttt{groups},
using \texttt{bnf\_names} as the keys. We'll represent a group with a
\texttt{list}, since we can easily append new members to the group. To
split our \texttt{scripts} into groups by
\texttt{\textquotesingle{}bnf\_name\textquotesingle{}}, we should
iterate over \texttt{scripts}, appending prescription dictionaries to
each group as we encounter them.

    \begin{Verbatim}[commandchars=\\\{\}]
{\color{incolor}In [{\color{incolor}13}]:} \PY{n}{groups} \PY{o}{=} \PY{p}{\PYZob{}}\PY{n}{name}\PY{p}{:} \PY{p}{[}\PY{p}{]} \PY{k}{for} \PY{n}{name} \PY{o+ow}{in} \PY{n}{bnf\PYZus{}names}\PY{p}{\PYZcb{}}
         \PY{k}{for} \PY{n}{script} \PY{o+ow}{in} \PY{n}{scripts}\PY{p}{:}
             \PY{n}{groups}\PY{p}{[}\PY{n}{script}\PY{p}{[}\PY{l+s+s1}{\PYZsq{}}\PY{l+s+s1}{bnf\PYZus{}name}\PY{l+s+s1}{\PYZsq{}}\PY{p}{]}\PY{p}{]}\PY{o}{.}\PY{n}{append}\PY{p}{(}\PY{n}{script}\PY{p}{)} 
\end{Verbatim}


    \begin{Verbatim}[commandchars=\\\{\}]
{\color{incolor}In [{\color{incolor}14}]:} \PY{c+c1}{\PYZsh{}for script in scripts:}
         \PY{c+c1}{\PYZsh{}   print(script[\PYZsq{}bnf\PYZus{}name\PYZsq{}])}
\end{Verbatim}


    Now that we've constructed our groups we should sum up
\texttt{\textquotesingle{}items\textquotesingle{}} in each group and
find the \texttt{\textquotesingle{}bnf\_name\textquotesingle{}} with the
largest sum. The result, \texttt{max\_item}, should have the form
\texttt{{[}(bnf\_name,\ item\ total){]}}, e.g.
\texttt{{[}(\textquotesingle{}Foobar\textquotesingle{},\ 2000){]}}.

    \begin{Verbatim}[commandchars=\\\{\}]
{\color{incolor}In [{\color{incolor}15}]:} \PY{n}{max\PYZus{}item} \PY{o}{=} \PY{p}{[}\PY{p}{]}
         \PY{n}{item\PYZus{}values} \PY{o}{=} \PY{p}{[}\PY{p}{]}
         \PY{n}{item\PYZus{}total} \PY{o}{=} \PY{l+m+mi}{0}
         \PY{n}{tuple\PYZus{}list} \PY{o}{=} \PY{p}{[}\PY{p}{]}
         
         \PY{k}{for} \PY{n}{k}\PY{p}{,} \PY{n}{v} \PY{o+ow}{in} \PY{n}{groups}\PY{o}{.}\PY{n}{items}\PY{p}{(}\PY{p}{)}\PY{p}{:}
             \PY{n}{item\PYZus{}name} \PY{o}{=} \PY{n}{k}
             \PY{n}{item\PYZus{}total} \PY{o}{=} \PY{l+m+mi}{0}
             \PY{n}{item\PYZus{}values} \PY{o}{=} \PY{n}{v}
             \PY{k}{for} \PY{n}{value} \PY{o+ow}{in} \PY{n}{item\PYZus{}values}\PY{p}{:}
                 \PY{n}{item\PYZus{}total} \PY{o}{=} \PY{n}{item\PYZus{}total} \PY{o}{+} \PY{n}{value}\PY{p}{[}\PY{l+s+s1}{\PYZsq{}}\PY{l+s+s1}{items}\PY{l+s+s1}{\PYZsq{}}\PY{p}{]}
                 
             \PY{n}{tuple\PYZus{}list}\PY{o}{.}\PY{n}{append}\PY{p}{(}\PY{n+nb}{tuple}\PY{p}{(}\PY{p}{[}\PY{n}{item\PYZus{}name}\PY{p}{,} \PY{n}{item\PYZus{}total}\PY{p}{]}\PY{p}{)}\PY{p}{)}
             
         \PY{n}{tuple\PYZus{}list}\PY{o}{.}\PY{n}{sort}\PY{p}{(}\PY{n}{key} \PY{o}{=} \PY{k}{lambda} \PY{n}{x}\PY{p}{:} \PY{n}{x}\PY{p}{[}\PY{l+m+mi}{1}\PY{p}{]}\PY{p}{)} 
         \PY{n}{max\PYZus{}item}\PY{o}{.}\PY{n}{append}\PY{p}{(}\PY{n}{tuple\PYZus{}list}\PY{p}{[}\PY{o}{\PYZhy{}}\PY{l+m+mi}{1}\PY{p}{]}\PY{p}{)}
         \PY{n}{max\PYZus{}item}
             
                 
\end{Verbatim}


\begin{Verbatim}[commandchars=\\\{\}]
{\color{outcolor}Out[{\color{outcolor}15}]:} [('Omeprazole\_Cap E/C 20mg', 113826)]
\end{Verbatim}
            
    \textbf{TIP:} If you are getting an error from the grader below, please
make sure your answer conforms to the correct format of
\texttt{{[}(bnf\_name,\ item\ total){]}}.

    \begin{Verbatim}[commandchars=\\\{\}]
{\color{incolor}In [{\color{incolor}16}]:} \PY{n}{grader}\PY{o}{.}\PY{n}{score}\PY{o}{.}\PY{n}{pw\PYZus{}\PYZus{}most\PYZus{}common\PYZus{}item}\PY{p}{(}\PY{n}{max\PYZus{}item}\PY{p}{)}
\end{Verbatim}


    \begin{Verbatim}[commandchars=\\\{\}]
==================
Your score:  1.0
==================

    \end{Verbatim}

    \textbf{Challenge:} Write a function that constructs groups as we did
above. The function should accept a list of dictionaries (e.g.
\texttt{scripts} or \texttt{practices}) and a tuple of fields to
\texttt{groupby} (e.g.
\texttt{(\textquotesingle{}bnf\_name\textquotesingle{})} or
\texttt{(\textquotesingle{}bnf\_name\textquotesingle{},\ \textquotesingle{}post\_code\textquotesingle{})})
and returns a dictionary of groups. The following questions will require
you to aggregate data in groups, so this could be a useful function for
the rest of the miniproject.

    \begin{Verbatim}[commandchars=\\\{\}]
{\color{incolor}In [{\color{incolor}28}]:} \PY{k}{def} \PY{n+nf}{group\PYZus{}by\PYZus{}field}\PY{p}{(}\PY{n}{data}\PY{p}{,} \PY{n}{fields}\PY{p}{)}\PY{p}{:}
             \PY{n}{new\PYZus{}dict} \PY{o}{=} \PY{p}{\PYZob{}}\PY{p}{\PYZcb{}} 
             \PY{n}{groupby\PYZus{}titles} \PY{o}{=} \PY{n+nb}{dict}\PY{p}{(}\PY{n}{fields}\PY{p}{)}
         
         \PY{c+c1}{\PYZsh{} groups = \PYZob{}name: [] for name in bnf\PYZus{}names\PYZcb{} or script in scripts:}
         \PY{c+c1}{\PYZsh{} groups[script[\PYZsq{}bnf\PYZus{}name\PYZsq{}]].append(script) }
\end{Verbatim}


    https://www.saltycrane.com/blog/2014/10/example-using-groupby-and-defaultdict-do-same-task/

    \begin{Verbatim}[commandchars=\\\{\}]
{\color{incolor}In [{\color{incolor} }]:} \PY{n}{groups} \PY{o}{=} \PY{n}{group\PYZus{}by\PYZus{}field}\PY{p}{(}\PY{n}{scripts}\PY{p}{,} \PY{p}{(}\PY{l+s+s1}{\PYZsq{}}\PY{l+s+s1}{bnf\PYZus{}name}\PY{l+s+s1}{\PYZsq{}}\PY{p}{,}\PY{p}{)}\PY{p}{)}
        \PY{n}{test\PYZus{}max\PYZus{}item} \PY{o}{=} \PY{n+nb}{max}\PY{p}{(}\PY{n}{group}\PY{p}{[}\PY{l+s+s1}{\PYZsq{}}\PY{l+s+s1}{item}\PY{l+s+s1}{\PYZsq{}}\PY{p}{]} \PY{k}{for} \PY{n}{group} \PY{o+ow}{in} \PY{n}{groups}\PY{p}{)}
        
        \PY{k}{assert} \PY{n}{test\PYZus{}max\PYZus{}item} \PY{o}{==} \PY{n}{max\PYZus{}item}
\end{Verbatim}


    \hypertarget{question-3-postal_totals}{%
\subsection{Question 3: postal\_totals}\label{question-3-postal_totals}}

Our data set is broken up among different files. This is typical for
tabular data to reduce redundancy. Each table typically contains data
about a particular type of event, processes, or physical object. Data on
prescriptions and medical practices are in separate files in our case.
If we want to find the total items prescribed in each postal code, we
will have to \emph{join} our prescription data (\texttt{scripts}) to our
clinic data (\texttt{practices}).

Find the total items prescribed in each postal code, representing the
results as a list of tuples
\texttt{(post\ code,\ total\ items\ prescribed)}. Sort your results
ascending alphabetically by post code and take only results from the
first 100 post codes. Only include post codes if there is at least one
prescription from a practice in that post code.

\textbf{NOTE:} Some practices have multiple postal codes associated with
them. Use the alphabetically first postal code.

    We can join \texttt{scripts} and \texttt{practices} based on the fact
that \texttt{\textquotesingle{}practice\textquotesingle{}} in
\texttt{scripts} matches
\texttt{\textquotesingle{}code\textquotesingle{}} in \texttt{practices}.
However, we must first deal with the repeated values of
\texttt{\textquotesingle{}code\textquotesingle{}} in \texttt{practices}.
We want the alphabetically first postal codes.

    \begin{Verbatim}[commandchars=\\\{\}]
{\color{incolor}In [{\color{incolor}8}]:} \PY{k+kn}{from} \PY{n+nn}{collections} \PY{k+kn}{import} \PY{n}{defaultdict}
        \PY{n}{group\PYZus{}by} \PY{o}{=} \PY{n}{defaultdict}\PY{p}{(}\PY{n+nb}{list}\PY{p}{)}
        \PY{k}{for} \PY{n}{practice} \PY{o+ow}{in} \PY{n}{practices}\PY{p}{:}
            \PY{c+c1}{\PYZsh{}for key,value in practice.items():}
                \PY{n}{key\PYZus{}string} \PY{o}{=} \PY{n}{practice}\PY{p}{[}\PY{l+s+s1}{\PYZsq{}}\PY{l+s+s1}{code}\PY{l+s+s1}{\PYZsq{}}\PY{p}{]}
                \PY{n}{mylist} \PY{o}{=} \PY{p}{[}\PY{p}{]}
                \PY{n}{mylist}\PY{o}{.}\PY{n}{append}\PY{p}{(}\PY{n}{practice}\PY{p}{)}
                \PY{n}{group\PYZus{}by}\PY{p}{[}\PY{n}{key\PYZus{}string}\PY{p}{]}\PY{o}{.}\PY{n}{append}\PY{p}{(}\PY{n}{mylist}\PY{p}{)}
            
        
            
        \PY{n}{practice\PYZus{}postal} \PY{o}{=} \PY{n+nb}{dict}\PY{p}{(}\PY{n}{group\PYZus{}by}\PY{p}{)}
        \PY{k}{for} \PY{n}{code} \PY{o+ow}{in} \PY{n}{practice\PYZus{}postal}\PY{o}{.}\PY{n}{keys}\PY{p}{(}\PY{p}{)}\PY{p}{:}
                \PY{k}{for} \PY{n}{i} \PY{o+ow}{in} \PY{n+nb}{range}\PY{p}{(}\PY{n+nb}{len}\PY{p}{(}\PY{n}{practice\PYZus{}postal}\PY{p}{[}\PY{n}{code}\PY{p}{]}\PY{p}{)}\PY{p}{)}\PY{p}{:}
                    \PY{k}{if} \PY{n}{practice\PYZus{}postal}\PY{p}{[}\PY{n}{code}\PY{p}{]}\PY{p}{[}\PY{l+m+mi}{0}\PY{p}{]}\PY{p}{[}\PY{l+m+mi}{0}\PY{p}{]}\PY{p}{[}\PY{l+s+s1}{\PYZsq{}}\PY{l+s+s1}{post\PYZus{}code}\PY{l+s+s1}{\PYZsq{}}\PY{p}{]} \PY{o}{==} \PY{n+nb}{min}\PY{p}{(}\PY{p}{(}\PY{n+nb}{str}\PY{p}{(}\PY{n}{practice\PYZus{}postal}\PY{p}{[}\PY{n}{code}\PY{p}{]}\PY{p}{[}\PY{l+m+mi}{0}\PY{p}{]}\PY{p}{[}\PY{l+m+mi}{0}\PY{p}{]}\PY{p}{[}\PY{l+s+s1}{\PYZsq{}}\PY{l+s+s1}{post\PYZus{}code}\PY{l+s+s1}{\PYZsq{}}\PY{p}{]}\PY{p}{)}\PY{p}{)}\PY{p}{,} \PY{n+nb}{str}\PY{p}{(}\PY{n}{practice\PYZus{}postal}\PY{p}{[}\PY{n}{code}\PY{p}{]}\PY{p}{[}\PY{n}{i}\PY{p}{]}\PY{p}{[}\PY{l+m+mi}{0}\PY{p}{]}\PY{p}{[}\PY{l+s+s1}{\PYZsq{}}\PY{l+s+s1}{post\PYZus{}code}\PY{l+s+s1}{\PYZsq{}}\PY{p}{]}\PY{p}{)}\PY{p}{)}\PY{p}{:}
                        \PY{k+kc}{None}
                    \PY{k}{else}\PY{p}{:}
                        \PY{n}{practice\PYZus{}postal}\PY{p}{[}\PY{n}{code}\PY{p}{]}\PY{p}{[}\PY{l+m+mi}{0}\PY{p}{]}\PY{p}{[}\PY{l+m+mi}{0}\PY{p}{]}\PY{o}{.}\PY{n}{update}\PY{p}{(}\PY{n}{practice\PYZus{}postal}\PY{p}{[}\PY{n}{code}\PY{p}{]}\PY{p}{[}\PY{n}{i}\PY{p}{]}\PY{p}{[}\PY{l+m+mi}{0}\PY{p}{]}\PY{p}{)}
        
        \PY{k}{for} \PY{n}{key}\PY{p}{,}\PY{n}{value} \PY{o+ow}{in} \PY{n}{practice\PYZus{}postal}\PY{o}{.}\PY{n}{items}\PY{p}{(}\PY{p}{)}\PY{p}{:}
            \PY{k}{if} \PY{n}{key} \PY{o}{==} \PY{l+s+s1}{\PYZsq{}}\PY{l+s+s1}{0}\PY{l+s+s1}{\PYZsq{}}\PY{p}{:}
                \PY{n+nb}{print}\PY{p}{(}\PY{l+s+s2}{\PYZdq{}}\PY{l+s+s2}{there is  0 value}\PY{l+s+s2}{\PYZdq{}}\PY{p}{)}
            \PY{k}{else}\PY{p}{:} 
                \PY{n}{practice\PYZus{}postal}\PY{p}{[}\PY{n}{key}\PY{p}{]} \PY{o}{=} \PY{n}{value}\PY{p}{[}\PY{l+m+mi}{0}\PY{p}{]}\PY{p}{[}\PY{l+m+mi}{0}\PY{p}{]}\PY{p}{[}\PY{l+s+s1}{\PYZsq{}}\PY{l+s+s1}{post\PYZus{}code}\PY{l+s+s1}{\PYZsq{}}\PY{p}{]}  
                
        \PY{n}{practice\PYZus{}postal}\PY{p}{[}\PY{l+s+s1}{\PYZsq{}}\PY{l+s+s1}{K82019}\PY{l+s+s1}{\PYZsq{}}\PY{p}{]}
                
\end{Verbatim}


\begin{Verbatim}[commandchars=\\\{\}]
{\color{outcolor}Out[{\color{outcolor}8}]:} 'HP21 8TR'
\end{Verbatim}
            
    https://stackoverflow.com/questions/13264511/typeerror-unhashable-type-dict

    \#for key,value in practice\_postal.items(): if key == `0':
print(``there is 0 value'') else: \# practice\_postal{[}key{]} =
value{[}0{]}\\
\#practice\_postal{[}`K82019'{]} \#group\_by{[}`K82019'{]}\\
test\_pref = `K82019' \#for practice in practices: res = \{key:val for
key, val in group\_by.items() if key.startswith(test\_pref)\}

print(``Filtered dictionary keys are :'' + str(res))

test\_pref = `K82019' for practice in practices: for k,v in
practice.items(): if practice{[}k{]} == `K82019':
print(``------------------------'') print(practice)

    \textbf{Challenge:} This is an aggregation of the practice data grouped
by practice codes. Write an alternative implementation of the above cell
using the \texttt{group\_by\_field} function you defined previously.

    \begin{Verbatim}[commandchars=\\\{\}]
{\color{incolor}In [{\color{incolor}9}]:} \PY{k}{assert} \PY{n}{practice\PYZus{}postal}\PY{p}{[}\PY{l+s+s1}{\PYZsq{}}\PY{l+s+s1}{K82019}\PY{l+s+s1}{\PYZsq{}}\PY{p}{]} \PY{o}{==} \PY{l+s+s1}{\PYZsq{}}\PY{l+s+s1}{HP21 8TR}\PY{l+s+s1}{\PYZsq{}}
\end{Verbatim}


    Now we can join \texttt{practice\_postal} to \texttt{scripts}.

    \begin{Verbatim}[commandchars=\\\{\}]
{\color{incolor}In [{\color{incolor}10}]:} \PY{n}{joined} \PY{o}{=} \PY{n}{scripts}\PY{p}{[}\PY{p}{:}\PY{p}{]}
         \PY{k}{for} \PY{n}{abc} \PY{o+ow}{in} \PY{n}{joined}\PY{p}{:}
             \PY{n}{abc}\PY{p}{[}\PY{l+s+s1}{\PYZsq{}}\PY{l+s+s1}{post\PYZus{}code}\PY{l+s+s1}{\PYZsq{}}\PY{p}{]} \PY{o}{=} \PY{n}{practice\PYZus{}postal}\PY{p}{[}\PY{n}{abc}\PY{p}{[}\PY{l+s+s1}{\PYZsq{}}\PY{l+s+s1}{practice}\PY{l+s+s1}{\PYZsq{}}\PY{p}{]}\PY{p}{]}
              
\end{Verbatim}


    \begin{Verbatim}[commandchars=\\\{\}]
{\color{incolor}In [{\color{incolor}11}]:} \PY{n}{postjoinedcode} \PY{o}{=} \PY{p}{[}\PY{n}{joined}\PY{p}{[}\PY{n}{i}\PY{p}{]}\PY{p}{[}\PY{l+s+s1}{\PYZsq{}}\PY{l+s+s1}{post\PYZus{}code}\PY{l+s+s1}{\PYZsq{}}\PY{p}{]} \PY{k}{for} \PY{n}{i} \PY{o+ow}{in} \PY{n+nb}{range}\PY{p}{(}\PY{n+nb}{len}\PY{p}{(}\PY{n}{joined}\PY{p}{)}\PY{p}{)}\PY{p}{]}
         \PY{n}{newgroups} \PY{o}{=} \PY{p}{\PYZob{}}\PY{n}{name}\PY{p}{:} \PY{p}{[}\PY{p}{]} \PY{k}{for} \PY{n}{name} \PY{o+ow}{in} \PY{n}{postjoinedcode}\PY{p}{\PYZcb{}}
         \PY{k}{for} \PY{n}{abc} \PY{o+ow}{in} \PY{n}{joined}\PY{p}{:}
             \PY{n}{newgroups}\PY{p}{[}\PY{n}{abc}\PY{p}{[}\PY{l+s+s1}{\PYZsq{}}\PY{l+s+s1}{post\PYZus{}code}\PY{l+s+s1}{\PYZsq{}}\PY{p}{]}\PY{p}{]}\PY{o}{.}\PY{n}{append}\PY{p}{(}\PY{n}{abc}\PY{p}{)} 
\end{Verbatim}


    Finally we'll group the prescription dictionaries in \texttt{joined} by
\texttt{\textquotesingle{}post\_code\textquotesingle{}} and sum up the
items prescribed in each group, as we did in the previous question.

    \begin{Verbatim}[commandchars=\\\{\}]
{\color{incolor}In [{\color{incolor}12}]:} \PY{n}{item\PYZus{}values} \PY{o}{=} \PY{p}{[}\PY{p}{]}
         
         \PY{n}{tuple\PYZus{}list} \PY{o}{=} \PY{p}{[}\PY{p}{]}
         
         \PY{k}{for} \PY{n}{k}\PY{p}{,} \PY{n}{v} \PY{o+ow}{in} \PY{n}{newgroups}\PY{o}{.}\PY{n}{items}\PY{p}{(}\PY{p}{)}\PY{p}{:}
             \PY{n}{item\PYZus{}total} \PY{o}{=} \PY{l+m+mi}{0}
             \PY{n}{postcode} \PY{o}{=} \PY{n}{k}
             \PY{n}{item\PYZus{}values} \PY{o}{=} \PY{n}{v}
             \PY{k}{for} \PY{n}{value} \PY{o+ow}{in} \PY{n}{item\PYZus{}values}\PY{p}{:}
                 \PY{n}{item\PYZus{}total} \PY{o}{=} \PY{n}{item\PYZus{}total} \PY{o}{+} \PY{n}{value}\PY{p}{[}\PY{l+s+s1}{\PYZsq{}}\PY{l+s+s1}{items}\PY{l+s+s1}{\PYZsq{}}\PY{p}{]}
                 
             \PY{n}{tuple\PYZus{}list}\PY{o}{.}\PY{n}{append}\PY{p}{(}\PY{n+nb}{tuple}\PY{p}{(}\PY{p}{[}\PY{n}{postcode}\PY{p}{,} \PY{n}{item\PYZus{}total}\PY{p}{]}\PY{p}{)}\PY{p}{)}
             
         \PY{n}{tuple\PYZus{}list}\PY{o}{.}\PY{n}{sort}\PY{p}{(}\PY{n}{key} \PY{o}{=} \PY{k}{lambda} \PY{n}{x}\PY{p}{:} \PY{n}{x}\PY{p}{[}\PY{l+m+mi}{0}\PY{p}{]}\PY{p}{)} 
         
         \PY{n}{my\PYZus{}100\PYZus{}list} \PY{o}{=} \PY{n}{tuple\PYZus{}list}\PY{p}{[}\PY{p}{:}\PY{l+m+mi}{100}\PY{p}{]}
                    
                     
\end{Verbatim}


    \begin{Verbatim}[commandchars=\\\{\}]
{\color{incolor}In [{\color{incolor}25}]:} \PY{n}{postal\PYZus{}totals} \PY{o}{=} \PY{n}{my\PYZus{}100\PYZus{}list}
         
         \PY{n}{grader}\PY{o}{.}\PY{n}{score}\PY{o}{.}\PY{n}{pw\PYZus{}\PYZus{}postal\PYZus{}totals}\PY{p}{(}\PY{n}{postal\PYZus{}totals}\PY{p}{)}
\end{Verbatim}


    \begin{Verbatim}[commandchars=\\\{\}]
==================
Your score:  1.0
==================

    \end{Verbatim}

    \hypertarget{question-4-items_by_region}{%
\subsection{Question 4:
items\_by\_region}\label{question-4-items_by_region}}

Now we'll combine the techniques we've developed to answer a more
complex question. Find the most commonly dispensed item in each postal
code, representing the results as a list of tuples (\texttt{post\_code},
\texttt{bnf\_name}, amount dispensed as proportion of total). Sort your
results ascending alphabetically by post code and take only results from
the first 100 post codes.

\textbf{NOTE:} We'll continue to use the \texttt{joined} variable we
created before, where we've chosen the alphabetically first postal code
for each practice. Additionally, some postal codes will have multiple
\texttt{\textquotesingle{}bnf\_name\textquotesingle{}} with the same
number of items prescribed for the maximum. In this case, we'll take the
alphabetically first
\texttt{\textquotesingle{}bnf\_name\textquotesingle{}}.

    Now we need to calculate the total items of each
\texttt{\textquotesingle{}bnf\_name\textquotesingle{}} prescribed in
each \texttt{\textquotesingle{}post\_code\textquotesingle{}}. Use the
techniques we developed in the previous questions to calculate these
totals. You should have 141196
\texttt{(\textquotesingle{}post\_code\textquotesingle{},\ \textquotesingle{}bnf\_name\textquotesingle{})}
groups.

    \begin{Verbatim}[commandchars=\\\{\}]
{\color{incolor}In [{\color{incolor}13}]:} \PY{k+kn}{from} \PY{n+nn}{collections} \PY{k+kn}{import} \PY{n}{defaultdict}
         \PY{n}{total\PYZus{}items\PYZus{}by\PYZus{}bnf\PYZus{}post} \PY{o}{=} \PY{n}{defaultdict}\PY{p}{(}\PY{n+nb}{list}\PY{p}{)}
         
         \PY{k}{for} \PY{n}{script} \PY{o+ow}{in} \PY{n}{joined}\PY{p}{:}
             \PY{n}{total\PYZus{}items\PYZus{}by\PYZus{}bnf\PYZus{}post}\PY{p}{[}\PY{n}{script}\PY{p}{[}\PY{l+s+s1}{\PYZsq{}}\PY{l+s+s1}{post\PYZus{}code}\PY{l+s+s1}{\PYZsq{}}\PY{p}{]}\PY{p}{,} \PY{n}{script}\PY{p}{[}\PY{l+s+s1}{\PYZsq{}}\PY{l+s+s1}{bnf\PYZus{}name}\PY{l+s+s1}{\PYZsq{}}\PY{p}{]}\PY{p}{]}\PY{o}{.}\PY{n}{append}\PY{p}{(}\PY{n}{script}\PY{p}{)}
            
         \PY{n}{sorted\PYZus{}dict} \PY{o}{=} \PY{n+nb}{dict}\PY{p}{(}\PY{n+nb}{sorted}\PY{p}{(}\PY{n}{total\PYZus{}items\PYZus{}by\PYZus{}bnf\PYZus{}post}\PY{o}{.}\PY{n}{items}\PY{p}{(}\PY{p}{)}\PY{p}{)}\PY{p}{)}
\end{Verbatim}


    \begin{Verbatim}[commandchars=\\\{\}]
{\color{incolor}In [{\color{incolor}14}]:} \PY{n}{total\PYZus{}items\PYZus{}by\PYZus{}bnf\PYZus{}post} \PY{o}{=} \PY{n}{sorted\PYZus{}dict}
         
             
             
         \PY{k}{assert} \PY{n+nb}{len}\PY{p}{(}\PY{n}{total\PYZus{}items\PYZus{}by\PYZus{}bnf\PYZus{}post}\PY{p}{)} \PY{o}{==} \PY{l+m+mi}{141196}
\end{Verbatim}


    Let's use \texttt{total\_items} to find the maximum item total for each
postal code. To do this, we will want to regroup
\texttt{total\_items\_by\_bnf\_post} by
\texttt{\textquotesingle{}post\_code\textquotesingle{}} only, not by
\texttt{(\textquotesingle{}post\_code\textquotesingle{},\ \textquotesingle{}bnf\_name\textquotesingle{})}.
First let's turn \texttt{total\_items} into a list of dictionaries
(similar to \texttt{scripts} or \texttt{practices}) and then group it by
\texttt{\textquotesingle{}post\_code\textquotesingle{}}. You should have
118 groups in the resulting \texttt{total\_items\_by\_post} after
grouping \texttt{total\_items} by
\texttt{\textquotesingle{}post\_code\textquotesingle{}}.

    \begin{Verbatim}[commandchars=\\\{\}]
{\color{incolor}In [{\color{incolor}15}]:} \PY{n}{total\PYZus{}items} \PY{o}{=} \PY{p}{[}\PY{p}{]}
         \PY{n}{newdict} \PY{o}{=} \PY{n}{defaultdict}\PY{p}{(}\PY{n+nb}{list}\PY{p}{)}
         \PY{k}{for} \PY{n}{key}\PY{p}{,}\PY{n}{value} \PY{o+ow}{in} \PY{n}{total\PYZus{}items\PYZus{}by\PYZus{}bnf\PYZus{}post}\PY{o}{.}\PY{n}{items}\PY{p}{(}\PY{p}{)}\PY{p}{:}
             \PY{n}{newdict}\PY{p}{[}\PY{n}{key}\PY{p}{[}\PY{l+m+mi}{0}\PY{p}{]}\PY{p}{]}\PY{o}{.}\PY{n}{append}\PY{p}{(}\PY{n}{value}\PY{p}{)}
             
         
         \PY{c+c1}{\PYZsh{}print(\PYZdq{}here is my new dict\PYZdq{})}
         \PY{c+c1}{\PYZsh{}newdict}
                 
         \PY{k}{for} \PY{n}{key}\PY{p}{,} \PY{n}{value} \PY{o+ow}{in} \PY{n}{newdict}\PY{o}{.}\PY{n}{items}\PY{p}{(}\PY{p}{)}\PY{p}{:} 
             \PY{n}{l} \PY{o}{=} \PY{p}{\PYZob{}}\PY{p}{\PYZcb{}}
             \PY{n}{l}\PY{p}{[}\PY{n}{key}\PY{p}{]} \PY{o}{=} \PY{n}{value}
             \PY{n}{total\PYZus{}items}\PY{o}{.}\PY{n}{append}\PY{p}{(}\PY{n}{l}\PY{p}{)}
             
         
             
             
                 
         
             
         \PY{n}{total\PYZus{}items\PYZus{}by\PYZus{}post} \PY{o}{=} \PY{n}{total\PYZus{}items}    
         \PY{n+nb}{len}\PY{p}{(}\PY{n}{total\PYZus{}items}\PY{p}{)}
             
         \PY{c+c1}{\PYZsh{}assert len(total\PYZus{}items\PYZus{}by\PYZus{}post) == 118}
\end{Verbatim}


\begin{Verbatim}[commandchars=\\\{\}]
{\color{outcolor}Out[{\color{outcolor}15}]:} 118
\end{Verbatim}
            
    \begin{Verbatim}[commandchars=\\\{\}]
{\color{incolor}In [{\color{incolor}15}]:} \PY{k}{assert} \PY{n+nb}{len}\PY{p}{(}\PY{n}{total\PYZus{}items\PYZus{}by\PYZus{}post}\PY{p}{)} \PY{o}{==} \PY{l+m+mi}{118}
\end{Verbatim}


    Now we will aggregate the groups in \texttt{total\_by\_item\_post} to
create \texttt{max\_item\_by\_post}. Some
\texttt{\textquotesingle{}bnf\_name\textquotesingle{}} have the same
item total within a given postal code. Therefore, if more than one
\texttt{\textquotesingle{}bnf\_name\textquotesingle{}} has the maximum
item total in a given postal code, we'll take the alphabetically first
\texttt{\textquotesingle{}bnf\_name\textquotesingle{}}. We can do this
by \href{https://docs.python.org/2.7/howto/sorting.html}{sorting} each
group according to the item total and
\texttt{\textquotesingle{}bnf\_name\textquotesingle{}}.

    \begin{Verbatim}[commandchars=\\\{\}]
{\color{incolor}In [{\color{incolor}16}]:} \PY{n}{max\PYZus{}item\PYZus{}by\PYZus{}post} \PY{o}{=} \PY{p}{[}\PY{p}{]}
         
         \PY{n}{puple\PYZus{}list} \PY{o}{=} \PY{p}{[}\PY{p}{]}
         \PY{n}{count} \PY{o}{=} \PY{l+m+mi}{0}
         \PY{k}{for} \PY{n}{onedef\PYZus{}dict} \PY{o+ow}{in} \PY{n}{total\PYZus{}items}\PY{p}{:}
                 \PY{k}{for} \PY{n}{k}\PY{p}{,}\PY{n}{v} \PY{o+ow}{in} \PY{n}{onedef\PYZus{}dict}\PY{o}{.}\PY{n}{items}\PY{p}{(}\PY{p}{)}\PY{p}{:}
                     \PY{n}{item\PYZus{}total} \PY{o}{=} \PY{l+m+mi}{0}
                     \PY{n}{postcode}\PY{o}{=} \PY{n}{k}
                     \PY{k}{for} \PY{n}{value} \PY{o+ow}{in} \PY{n}{v}\PY{p}{:}
                         \PY{n}{item\PYZus{}total} \PY{o}{=} \PY{l+m+mi}{0}
                         \PY{c+c1}{\PYZsh{}print(\PYZdq{}the length of this v[1] is \PYZdq{} + str(len(v[1])))}
                         \PY{c+c1}{\PYZsh{}print(value)}
                         \PY{c+c1}{\PYZsh{}print(\PYZdq{}\PYZhy{}\PYZhy{}\PYZhy{}\PYZhy{}\PYZhy{}\PYZhy{}\PYZhy{}\PYZhy{}\PYZhy{}\PYZhy{}\PYZhy{}\PYZhy{}\PYZhy{}\PYZhy{}\PYZhy{}\PYZhy{}\PYZhy{}\PYZhy{}\PYZhy{}\PYZhy{}\PYZhy{}\PYZhy{}\PYZhy{}\PYZhy{}\PYZdq{}) }
                         
                         \PY{k}{for} \PY{n}{val} \PY{o+ow}{in} \PY{n}{value}\PY{p}{:}
                             
                             \PY{n}{item\PYZus{}total} \PY{o}{=} \PY{n}{item\PYZus{}total} \PY{o}{+} \PY{n+nb}{int}\PY{p}{(}\PY{n}{val}\PY{p}{[}\PY{l+s+s1}{\PYZsq{}}\PY{l+s+s1}{items}\PY{l+s+s1}{\PYZsq{}}\PY{p}{]}\PY{p}{)}
                             \PY{n}{bnf\PYZus{}name} \PY{o}{=} \PY{n}{val}\PY{p}{[}\PY{l+s+s1}{\PYZsq{}}\PY{l+s+s1}{bnf\PYZus{}name}\PY{l+s+s1}{\PYZsq{}}\PY{p}{]} 
                             \PY{c+c1}{\PYZsh{}print(\PYZdq{}the bnf name assigned is \PYZdq{} + val[\PYZsq{}bnf\PYZus{}name\PYZsq{}] )}
                         \PY{n}{l} \PY{o}{=} \PY{n+nb}{tuple}\PY{p}{(}\PY{p}{[}\PY{n}{postcode}\PY{p}{,} \PY{n}{bnf\PYZus{}name}\PY{p}{,} \PY{n}{item\PYZus{}total}\PY{p}{]}\PY{p}{)}
                             \PY{c+c1}{\PYZsh{}print (l)}
                         \PY{n}{puple\PYZus{}list}\PY{o}{.}\PY{n}{append}\PY{p}{(}\PY{n}{l}\PY{p}{)}
                 \PY{c+c1}{\PYZsh{}print(\PYZdq{}\PYZhy{}\PYZhy{}\PYZhy{}\PYZhy{}\PYZhy{}\PYZhy{}\PYZhy{}\PYZhy{}\PYZhy{}Every POstal change\PYZhy{}\PYZhy{}\PYZhy{}\PYZhy{}\PYZhy{}\PYZhy{}\PYZhy{}\PYZhy{}\PYZhy{}\PYZhy{}\PYZhy{}\PYZhy{}\PYZhy{}\PYZhy{}\PYZhy{}\PYZhy{}\PYZhy{}\PYZhy{}\PYZhy{}\PYZhy{}\PYZhy{}\PYZhy{}\PYZhy{}\PYZhy{}\PYZhy{}\PYZhy{}\PYZhy{} \PYZdq{})}
                 
         \PY{n}{puple\PYZus{}list}\PY{p}{[}\PY{p}{:}\PY{l+m+mi}{10}\PY{p}{]}  
         \PY{c+c1}{\PYZsh{}print(len(puple\PYZus{}list))}
         \PY{c+c1}{\PYZsh{}print(count)}
         \PY{c+c1}{\PYZsh{}print(len)}
\end{Verbatim}


\begin{Verbatim}[commandchars=\\\{\}]
{\color{outcolor}Out[{\color{outcolor}16}]:} [('B11 4BW', '3m Health Care\_Cavilon Durable Barrier C', 7),
          ('B11 4BW', '3m Health Care\_Cavilon No Sting Barrier', 2),
          ('B11 4BW', 'Abidec\_Dps', 63),
          ('B11 4BW', 'Acetic Acid\_Ear Spy 2\% 5ml', 9),
          ('B11 4BW', 'Aciclovir\_Crm 5\%', 13),
          ('B11 4BW', 'Aciclovir\_Tab 200mg', 4),
          ('B11 4BW', 'Aciclovir\_Tab 800mg', 9),
          ('B11 4BW', 'Aciclovir\_Tab Disper 400mg', 2),
          ('B11 4BW', 'Adalat LA 30\_Tab 30mg', 3),
          ('B11 4BW', 'Adalat LA 60\_Tab 60mg', 11)]
\end{Verbatim}
            
    In order to express the item totals as a proportion of the total amount
of items prescribed across all
\texttt{\textquotesingle{}bnf\_name\textquotesingle{}} in a postal code,
we'll need to use the total items prescribed that we previously
calculated as \texttt{items\_by\_post}. Calculate the proportions for
the most common \texttt{\textquotesingle{}bnf\_names\textquotesingle{}}
for each postal code. Format your answer as a list of tuples:
\texttt{{[}(post\_code,\ bnf\_name,\ total){]}}

    \begin{Verbatim}[commandchars=\\\{\}]
{\color{incolor}In [{\color{incolor}47}]:} \PY{k+kn}{from} \PY{n+nn}{operator} \PY{k+kn}{import} \PY{n}{itemgetter}\PY{p}{,} \PY{n}{attrgetter}
         
         \PY{n}{s} \PY{o}{=} \PY{n+nb}{sorted}\PY{p}{(}\PY{n}{puple\PYZus{}list}\PY{p}{,} \PY{n}{key}\PY{o}{=}\PY{n}{itemgetter}\PY{p}{(}\PY{l+m+mi}{2}\PY{p}{)}\PY{p}{,} \PY{n}{reverse}\PY{o}{=}\PY{k+kc}{True}\PY{p}{)}
         
         \PY{n}{finaltup} \PY{o}{=} \PY{n+nb}{sorted}\PY{p}{(}\PY{n}{s}\PY{p}{,} \PY{n}{key}\PY{o}{=}\PY{n}{itemgetter}\PY{p}{(}\PY{l+m+mi}{0}\PY{p}{)}\PY{p}{)}
         \PY{n}{prevtup} \PY{o}{=} \PY{n}{finaltup}\PY{p}{[}\PY{l+m+mi}{0}\PY{p}{]}\PY{p}{[}\PY{l+m+mi}{0}\PY{p}{]}
         \PY{n}{postsum} \PY{o}{=} \PY{l+m+mi}{0}
         \PY{n}{r} \PY{o}{=} \PY{l+m+mi}{0}
         \PY{n}{postalsumlist} \PY{o}{=} \PY{p}{[}\PY{p}{]}
         \PY{k}{for} \PY{n}{tup} \PY{o+ow}{in} \PY{n}{finaltup}\PY{p}{:}
             \PY{k}{if} \PY{n+nb}{str}\PY{p}{(}\PY{n}{tup}\PY{p}{[}\PY{l+m+mi}{0}\PY{p}{]}\PY{p}{)} \PY{o}{==} \PY{n+nb}{str}\PY{p}{(}\PY{n}{prevtup}\PY{p}{)}\PY{p}{:}
                 
                 \PY{n}{postsum} \PY{o}{=} \PY{n}{postsum} \PY{o}{+} \PY{n}{tup}\PY{p}{[}\PY{l+m+mi}{2}\PY{p}{]}
                 \PY{c+c1}{\PYZsh{}print(\PYZsq{}jndvc\PYZsq{})}
                 \PY{n}{prevtup} \PY{o}{=} \PY{n}{tup}\PY{p}{[}\PY{l+m+mi}{0}\PY{p}{]}
                 \PY{n}{r}\PY{o}{+}\PY{o}{=}\PY{l+m+mi}{1}
                 \PY{k}{if} \PY{n}{r} \PY{o}{==} \PY{p}{(}\PY{n+nb}{len}\PY{p}{(}\PY{n}{finaltup}\PY{p}{)}\PY{o}{\PYZhy{}}\PY{l+m+mi}{1}\PY{p}{)}\PY{p}{:}
                     \PY{n}{postalsumlist}\PY{o}{.}\PY{n}{append}\PY{p}{(}\PY{n}{postsum}\PY{p}{)} 
                     \PY{k}{break}
             \PY{k}{else}\PY{p}{:}
                 \PY{n}{r} \PY{o}{+}\PY{o}{=}\PY{l+m+mi}{1}
                 \PY{n}{postalsumlist}\PY{o}{.}\PY{n}{append}\PY{p}{(}\PY{n}{postsum}\PY{p}{)}
                 \PY{n}{postsum} \PY{o}{=} \PY{n}{tup}\PY{p}{[}\PY{l+m+mi}{2}\PY{p}{]}
                 \PY{n}{prevtup} \PY{o}{=} \PY{n}{tup}\PY{p}{[}\PY{l+m+mi}{0}\PY{p}{]}
                 \PY{c+c1}{\PYZsh{}print(\PYZsq{}jndvc\PYZsq{})}
                 \PY{k+kc}{None}
         
         \PY{n}{i}\PY{o}{=}\PY{l+m+mi}{0} 
         \PY{n}{prevtup} \PY{o}{=} \PY{n}{finaltup}\PY{p}{[}\PY{l+m+mi}{0}\PY{p}{]}\PY{p}{[}\PY{l+m+mi}{0}\PY{p}{]}
         \PY{n}{tupandtotallist} \PY{o}{=} \PY{p}{[}\PY{p}{]}
         \PY{k}{for} \PY{n}{tup} \PY{o+ow}{in} \PY{n}{finaltup}\PY{p}{:}
             \PY{k}{if} \PY{n+nb}{str}\PY{p}{(}\PY{n}{tup}\PY{p}{[}\PY{l+m+mi}{0}\PY{p}{]}\PY{p}{)} \PY{o}{==} \PY{n+nb}{str}\PY{p}{(}\PY{n}{prevtup}\PY{p}{)}\PY{p}{:}
                 \PY{n}{tupandtotallist}\PY{o}{.}\PY{n}{append}\PY{p}{(}\PY{n}{postalsumlist}\PY{p}{[}\PY{n}{i}\PY{p}{]}\PY{p}{)}
                 
             \PY{k}{else}\PY{p}{:}
                 \PY{n}{prevtup} \PY{o}{=} \PY{n}{tup}\PY{p}{[}\PY{l+m+mi}{0}\PY{p}{]}
                 \PY{n}{i} \PY{o}{+}\PY{o}{=}\PY{l+m+mi}{1}
                 \PY{n}{tupandtotallist}\PY{o}{.}\PY{n}{append}\PY{p}{(}\PY{n}{postalsumlist}\PY{p}{[}\PY{n}{i}\PY{p}{]}\PY{p}{)}
               
         \PY{n+nb}{print}\PY{p}{(}\PY{n}{postalsumlist}\PY{p}{[}\PY{l+m+mi}{0}\PY{p}{:}\PY{l+m+mi}{10}\PY{p}{]}\PY{p}{)}
         \PY{n}{i}\PY{o}{=} \PY{l+m+mi}{0}
         \PY{n}{ratiotup} \PY{o}{=} \PY{p}{[}\PY{p}{]}
         \PY{n}{prevtup} \PY{o}{=} \PY{n}{finaltup}\PY{p}{[}\PY{l+m+mi}{0}\PY{p}{]}
         \PY{k}{for} \PY{n}{tup} \PY{o+ow}{in} \PY{n}{finaltup}\PY{p}{:}
             \PY{n}{n} \PY{o}{=} \PY{n}{tup}\PY{p}{[}\PY{l+m+mi}{2}\PY{p}{]}\PY{o}{/}\PY{n}{tupandtotallist}\PY{p}{[}\PY{n}{i}\PY{p}{]}
             \PY{n}{i} \PY{o}{+}\PY{o}{=}\PY{l+m+mi}{1}
             \PY{n}{ratiotup}\PY{o}{.}\PY{n}{append}\PY{p}{(}\PY{n+nb}{tuple}\PY{p}{(}\PY{p}{[}\PY{n}{tup}\PY{p}{[}\PY{l+m+mi}{0}\PY{p}{]}\PY{p}{,} \PY{n}{tup}\PY{p}{[}\PY{l+m+mi}{1}\PY{p}{]}\PY{p}{,}  \PY{n}{n}\PY{p}{]}\PY{p}{)}\PY{p}{)}
             
         
         \PY{n}{counter} \PY{o}{=} \PY{l+m+mi}{0}
         \PY{n}{postlist} \PY{o}{=} \PY{p}{[}\PY{p}{]}
         \PY{n}{firstpost100} \PY{o}{=} \PY{p}{[}\PY{p}{]}
         \PY{k}{for} \PY{n}{tup} \PY{o+ow}{in} \PY{n}{ratiotup}\PY{p}{:}
             \PY{k}{if} \PY{n}{tup}\PY{p}{[}\PY{l+m+mi}{0}\PY{p}{]} \PY{o+ow}{not} \PY{o+ow}{in} \PY{n}{postlist} \PY{o+ow}{and} \PY{n}{counter} \PY{o}{\PYZlt{}}\PY{l+m+mi}{100}\PY{p}{:}
                 \PY{n}{firstpost100}\PY{o}{.}\PY{n}{append}\PY{p}{(}\PY{n}{tup}\PY{p}{)}
                 \PY{n}{postlist}\PY{o}{.}\PY{n}{append}\PY{p}{(}\PY{n}{tup}\PY{p}{[}\PY{l+m+mi}{0}\PY{p}{]}\PY{p}{)}
                 \PY{n}{counter} \PY{o}{+}\PY{o}{=}\PY{l+m+mi}{1}
                 
             \PY{k}{else}\PY{p}{:} 
                 \PY{k+kc}{None}
         
         \PY{n}{firstpost100}         
\end{Verbatim}


    \begin{Verbatim}[commandchars=\\\{\}]
[20673, 19001, 29103, 24859, 36531, 34356, 28254, 54514, 29388, 44585]

    \end{Verbatim}

\begin{Verbatim}[commandchars=\\\{\}]
{\color{outcolor}Out[{\color{outcolor}47}]:} [('B11 4BW', 'Salbutamol\_Inha 100mcg (200 D) CFF', 0.03415082474725487),
          ('B18 7AL', 'Salbutamol\_Inha 100mcg (200 D) CFF', 0.02926161780958897),
          ('B21 9RY', 'Metformin HCl\_Tab 500mg', 0.03549462254750369),
          ('B23 6DJ', 'Lansoprazole\_Cap 30mg (E/C Gran)', 0.024095900880968663),
          ('B70 7AW', 'Paracet\_Tab 500mg', 0.0266896608360023),
          ('BB11 2DL', 'Omeprazole\_Cap E/C 20mg', 0.02884503434625684),
          ('BB2 1AX', 'Omeprazole\_Cap E/C 20mg', 0.03645501521908402),
          ('BB3 1PY', 'Omeprazole\_Cap E/C 20mg', 0.03428477088454342),
          ('BB4 5SL', 'Omeprazole\_Cap E/C 20mg', 0.040696883081529876),
          ('BB7 2JG', 'Omeprazole\_Cap E/C 20mg', 0.029471795446899183),
          ('BB8 0JZ', 'Atorvastatin\_Tab 20mg', 0.022563442442074293),
          ('BB9 7SR', 'Omeprazole\_Cap E/C 20mg', 0.023833193804939305),
          ('BD3 8QH', 'Atorvastatin\_Tab 40mg', 0.03422179914326511),
          ('BH18 8EE', 'Omeprazole\_Cap E/C 20mg', 0.029000583563798747),
          ('BH23 3AF', 'Omeprazole\_Cap E/C 20mg', 0.03733292364418497),
          ('BL1 8TU', 'Omeprazole\_Cap E/C 20mg', 0.03095821215368131),
          ('BL3 5HP', 'Omeprazole\_Cap E/C 20mg', 0.03359487236158692),
          ('BL9 0NJ', 'Omeprazole\_Cap E/C 20mg', 0.033622356683924895),
          ('BL9 0SN', 'Omeprazole\_Cap E/C 20mg', 0.03257264351523742),
          ('CB9 8HF', 'Omeprazole\_Cap E/C 20mg', 0.03586107485828934),
          ('CH1 4DS', 'Lansoprazole\_Cap 30mg (E/C Gran)', 0.026979808105398826),
          ('CH65 6TG', 'Lansoprazole\_Cap 30mg (E/C Gran)', 0.027421283379832604),
          ('CT11 8AD', 'Omeprazole\_Cap E/C 20mg', 0.02371612786870463),
          ('CV1 4FS', 'Omeprazole\_Cap E/C 20mg', 0.02988443966675625),
          ('CW1 3AW', 'Omeprazole\_Cap E/C 20mg', 0.03687757394234369),
          ('CW5 5NX', 'Omeprazole\_Cap E/C 20mg', 0.036574992911823076),
          ('CW7 1AT', 'Omeprazole\_Cap E/C 20mg', 0.038342136965990176),
          ('DA1 2HA', 'Omeprazole\_Cap E/C 20mg', 0.020977948226270374),
          ('DA11 8BZ', 'Amoxicillin\_Cap 500mg', 0.021502698215026983),
          ('DN22 7XF', 'Simvastatin\_Tab 40mg', 0.019888143695899377),
          ('DN34 4GB', 'Omeprazole\_Cap E/C 20mg', 0.03894778497490263),
          ('FY2 0JG', 'Omeprazole\_Cap E/C 20mg', 0.03794959344888452),
          ('FY4 1TJ', 'Omeprazole\_Cap E/C 20mg', 0.04512928155710333),
          ('FY5 2TZ', 'Omeprazole\_Cap E/C 20mg', 0.037575127660535945),
          ('FY7 8GU', 'Omeprazole\_Cap E/C 20mg', 0.03420067879209816),
          ('GL1 3PX', 'Omeprazole\_Cap E/C 20mg', 0.027334732423924448),
          ('GL50 4DP', 'Omeprazole\_Cap E/C 20mg', 0.02469861805351367),
          ('GU9 9QS', 'Omeprazole\_Cap E/C 20mg', 0.028601661946406898),
          ('HA0 4UZ', 'Metformin HCl\_Tab 500mg', 0.02961986376620523),
          ('HA3 7LT', 'Omeprazole\_Cap E/C 20mg', 0.0265001712702021),
          ('HG1 5AR', 'Omeprazole\_Cap E/C 20mg', 0.030014686084934523),
          ('HU7 4DW', 'Salbutamol\_Inha 100mcg (200 D) CFF', 0.026594986458142424),
          ('KT14 6DH', 'Amlodipine\_Tab 5mg', 0.01980716047537185),
          ('KT6 6EZ', 'Omeprazole\_Cap E/C 20mg', 0.029557408595253368),
          ('L31 0DJ', 'Omeprazole\_Cap E/C 20mg', 0.031093092156556992),
          ('L36 7XY', 'Salbutamol\_Inha 100mcg (200 D) CFF', 0.0304811669932506),
          ('L5 0QW', 'Omeprazole\_Cap E/C 20mg', 0.03383854757659264),
          ('L7 6HD', 'Salbutamol\_Inha 100mcg (200 D) CFF', 0.026874016302943456),
          ('LA1 1PN', 'Omeprazole\_Cap E/C 20mg', 0.03593535438892997),
          ('LE10 1DS', 'Aspirin Disper\_Tab 75mg', 0.02211411776629168),
          ('LE18 2EW', 'Lansoprazole\_Cap 15mg (E/C Gran)', 0.026545337066551798),
          ('LE5 3GH', 'Metformin HCl\_Tab 500mg', 0.0354924268862986),
          ('LN2 2JP', 'Omeprazole\_Cap E/C 20mg', 0.036189981157819504),
          ('LS9 9EF', 'Lansoprazole\_Cap 30mg (E/C Gran)', 0.03184116875819442),
          ('M11 4EJ', 'Omeprazole\_Cap E/C 20mg', 0.03207286540619874),
          ('M26 2SP', 'Omeprazole\_Cap E/C 20mg', 0.04085582480513283),
          ('M30 0NU', 'Omeprazole\_Cap E/C 20mg', 0.03824666953158573),
          ('M35 0AD', 'Omeprazole\_Cap E/C 20mg', 0.03085140306122449),
          ('ME8 8AA', 'Omeprazole\_Cap E/C 20mg', 0.02328067466444946),
          ('N9 7HD', 'Lansoprazole\_Cap 30mg (E/C Gran)', 0.026281508499967434),
          ('NE10 9QG', 'Paracet\_Tab 500mg', 0.044305701820370094),
          ('NE24 1DX', 'Paracet\_Tab 500mg', 0.0320651205165277),
          ('NE37 2PU', 'Paracet\_Tab 500mg', 0.02777391304347826),
          ('NE38 7NQ', 'Paracet\_Tab 500mg', 0.031134594625305382),
          ('NG7 3GW', 'Paracet\_Tab 500mg', 0.022318906091083737),
          ('NG7 5HY', 'Paracet\_Tab 500mg', 0.02509737782596474),
          ('NN16 8DN', 'Omeprazole\_Cap E/C 20mg', 0.03547300624372181),
          ('NW10 8RY', 'Metformin HCl\_Tab 500mg', 0.030900570686215375),
          ('OL1 1NL', 'Omeprazole\_Cap E/C 20mg', 0.033425912390975976),
          ('OL11 1DN', 'Omeprazole\_Cap E/C 20mg', 0.032178791672462556),
          ('OL4 1YN', 'Omeprazole\_Cap E/C 20mg', 0.02989427633977397),
          ('OL9 7AY', 'Omeprazole\_Cap E/C 20mg', 0.03317602310347256),
          ('PL7 1AD', 'Omeprazole\_Cap E/C 20mg', 0.03856973759050591),
          ('RM3 9SU', 'Salbutamol\_Inha 100mcg (200 D) CFF', 0.027516093578269743),
          ('S63 9EH', 'Salbutamol\_Inha 100mcg (200 D) CFF', 0.03003995745537126),
          ('S65 1DA', 'Influenza\_Vac Inact 0.5ml Pfs', 0.023771326534882748),
          ('S74 9AF', 'Omeprazole\_Cap E/C 20mg', 0.035946116987614136),
          ('SE1 6JP', 'Metformin HCl\_Tab 500mg', 0.023339995558516544),
          ('SE15 5LJ', 'Metformin HCl\_Tab 500mg', 0.02451173155066195),
          ('SK11 6JL', 'Omeprazole\_Cap E/C 20mg', 0.029244760200234393),
          ('SK6 1ND', 'Lansoprazole\_Cap 30mg (E/C Gran)', 0.019072510860735352),
          ('SM3 8EP', 'Omeprazole\_Cap E/C 20mg', 0.02976166633286601),
          ('SM6 0HY', 'Omeprazole\_Cap E/C 20mg', 0.03170636300417246),
          ('SR4 7XF', 'Paracet\_Tab 500mg', 0.03414722227795277),
          ('SR5 2LT', 'Paracet\_Tab 500mg', 0.03603964806771355),
          ('SS0 7AF', 'Omeprazole\_Cap E/C 20mg', 0.02118781584681719),
          ('SS13 3HQ', 'Lansoprazole\_Cap 30mg (E/C Gran)', 0.025575779606885986),
          ('SS8 0JA', 'Omeprazole\_Cap E/C 20mg', 0.026893212353864944),
          ('SS9 5UU', 'Influvac Sub-Unit\_Vac 0.5ml Pfs', 0.02536884460624641),
          ('ST1 4PB', 'Omeprazole\_Cap E/C 20mg', 0.029595079869567012),
          ('ST3 6AB', 'Omeprazole\_Cap E/C 20mg', 0.03128794729363132),
          ('ST8 6AG', 'Omeprazole\_Cap E/C 20mg', 0.03963379302352532),
          ('TN24 0GP', 'Amoxicillin\_Cap 500mg', 0.08493069890887643),
          ('TN34 1BA', 'Omeprazole\_Cap E/C 20mg', 0.03102866779089376),
          ('TR1 2JA', 'Omeprazole\_Cap E/C 20mg', 0.04630044285216699),
          ('TS1 2NX', 'Paracet\_Tab 500mg', 0.027549713373789975),
          ('TS10 4NW', 'Paracet\_Tab 500mg', 0.023228006465755853),
          ('TS17 0EE', 'Lansoprazole\_Cap 30mg (E/C Gran)', 0.025048254079663098),
          ('TS23 2DG', 'Paracet\_Tab 500mg', 0.025532452758642483),
          ('TS24 7PW', 'Paracet\_Tab 500mg', 0.03858642431322693)]
\end{Verbatim}
            
    \begin{Verbatim}[commandchars=\\\{\}]
{\color{incolor}In [{\color{incolor}48}]:} \PY{n}{items\PYZus{}by\PYZus{}region} \PY{o}{=} \PY{n}{firstpost100}\PY{p}{[}\PY{p}{:}\PY{l+m+mi}{100}\PY{p}{]}
\end{Verbatim}


    \begin{Verbatim}[commandchars=\\\{\}]
{\color{incolor}In [{\color{incolor}49}]:} \PY{n}{grader}\PY{o}{.}\PY{n}{score}\PY{o}{.}\PY{n}{pw\PYZus{}\PYZus{}items\PYZus{}by\PYZus{}region}\PY{p}{(}\PY{n}{items\PYZus{}by\PYZus{}region}\PY{p}{)}
\end{Verbatim}


    \begin{Verbatim}[commandchars=\\\{\}]
==================
Your score:  1.0
==================

    \end{Verbatim}

    \emph{Copyright © 2020 The Data Incubator. All rights reserved.}


    % Add a bibliography block to the postdoc
    
    
    
    \end{document}
